% vim:ts=4:sw=4
% Copyright (c) 2014 Casper Ti. Vector
% Public domain.

\begin{cabstract}
	% 中文测试文字。

		交通问题是当今世界关注的热点问题。并且,随着人们生活水平的提高、交通系统的发展,社会对交通的需求也日益增大。交通与环境、交通与能源、交通与需求之间的矛盾日益加剧,交通事故和堵塞给人们带来了巨大的效率、能源和生命上的损失。在高速公路的建设中,如何选取关键路段,并对关键路段进行针对性建设已经成为一个热点问题。

		在传统交通关键路段研究中,研究方法主要有\ding{172}针对交通网络拓扑结构的研究;\ding{173}针对交通网络统计数据进行分析;\ding{174}在微观层面下,针对路段的特性进行研究。这些方法都有各自的局限性。方法\ding{172}只针对了路网的拓扑结构,没有考虑节点之间的信息交流;方法\ding{173}基于统计学方法,该方法只能静态分析路网关键路段,无法分析关键路段随着时间、路网流量的变化规律;方法\ding{174}主要研究交通的微观领域特性,没有考虑路网的宏观变化规律。在此我们提出一种基于宏观高速公路网络的目标模型。

		本文从智能交通的实际应用角度出发,针对现有关键路段挖掘方法的不足,提出两个研究方法:\ding{172}提出一种基于贪心算法的关键路段挖掘模型,并证明贪心算法的可行性;\ding{173}提出一种基于复杂网络社群划分的关键路段挖掘模型,实现动态挖掘关键路段。

		集成上述成果,实现了一个基于B/S架构的高速公路关键路段挖掘系统,并且在真实应用场景下初步验证了原型系统的可靠性和适用性。
\end{cabstract}

\begin{eabstract}
	Traffic problem is a hot issue in the world today. With the improvement of people's living standard and the development of traffic system, the demand for traffic is also increasing. The contradiction between traffic and environment, transportation and energy, traffic and demand is becoming more and more serious. Traffic accidents and congestion have brought great efficiency, energy and life losses. In the construction of expressway, how to select the key sections and carry out the targeted construction of key sections has become a hot issue.
	
	In the research section of the traditional traffic key, the main research methods are 1) for research of traffic network topology; 2) according to the analysis of network traffic statistical data; 3) at the micro level, to study the characteristics of road. Each of these methods has its limitations. Method 1) only for the topology of the network, do not consider the exchange of information between nodes; 2) method based on statistical methods, this method can only static analysis of key sections of road network, to analysis key sections changes with time / network flow method; 3) the micro field characteristics of main traffic, without considering the macro changes the road network. Here, we propose a target model based on macro freeway network.

	This paper from the perspective of practical application of intelligent transportation, the existing key sections of mining method, put forward two research methods: 1) put forward a mining model of key sections based on the greedy algorithm, and prove the feasibility of the greedy algorithm; 2) put forward a mining model of key sections of complicated network community based on partitioning, dynamic mining key sections.
	
	Integrating the above results, a freeway critical link mining system based on B/S architecture is implemented, and the reliability and applicability of the prototype system are preliminarily verified in real application scenarios.

\end{eabstract}

