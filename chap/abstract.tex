% vim:ts=4:sw=4
% Copyright (c) 2014 Casper Ti. Vector
% Public domain.

\begin{cabstract}
	% 中文测试文字。

		交通问题是当今世界关注的热点问题。随着人们生活水平的提高、交通系统的发展,社会对交通的需求也日益增大。交通与环境、交通与能源、交通与需求之间的矛盾日益加剧,交通事故和堵塞给人们带来了巨大的效率、能源和生命上的损失。在高速公路的建设中,如何选取关键路段,并对关键路段进行针对性建设已经成为一个热点问题。

		在传统交通关键路段研究中,研究方法主要有\ding{172}针对交通网络拓扑结构的研究;\ding{173}针对交通网络统计数据进行分析;\ding{174}在微观层面下,针对路段的特性进行研究。这些方法都有各自的局限性。方法\ding{172}只针对了路网的拓扑结构,没有考虑节点之间的信息交流;方法\ding{173}基于统计学方法,该方法只能静态分析路网关键路段,无法分析关键路段随着时间/路网流量的变化规律;方法\ding{174}主要研究交通的微观领域特性,没有考虑路网的宏观变化规律。在此我们提出一种基于宏观高速公路网络的目标模型。

		本文从智能交通的实际应用角度出发,针对现有关键路段挖掘方法的不足,提出两个研究方法:\ding{172}提出一种基于贪心算法的关键路段挖掘模型,并证明贪心算法的可行性;\ding{173}提出一种基于复杂网络社群划分的关键路段挖掘模型,实现动态挖掘关键路段。

		集成上述成果,实现了一个基于B/S架构的高速公路关键路段挖掘系统,并且在真实应用场景下初步验证了原型系统的可靠性和适用性。
\end{cabstract}

\begin{eabstract}
	Traffic problem is a hot issue in the world today. With the improvement of people's living standard and the development of the traffic system, the demand of social traffic is increasing day by day. The contradiction between traffic and environment, transportation and energy, traffic and demand increasing, traffic accidents and congestion brings efficiency, energy and life in the great loss, simple traffic control technology has been unable to meet the demand. The study of the traditional intelligent transportation is based on the research of the single spatial location in the road network, which is based on the theory of dynamics, statistics, simulation and machine learning. With the continuous development of the traffic system, traffic system gradually presents the network situation, the close connection between each node in the network, the macro characteristics of single point position is not enough to describe the whole Expressway network. With the traffic accident has become the bottleneck of the traffic system, the research of intelligent transportation system has new demand: how to find the key nodes in the traffic system, by processing the key nodes, in order to reduce the traffic paralysis rate, increase the operation stability of road network to.
	
	In the intelligent highway system network, the research focused on 1 key nodes) study on traffic network topology; 2) for statistical research on traffic network information; 3) using the propagation dynamics, the research of micro site. Our purpose is to find the key nodes in the network, improve the efficiency in the whole network, method 1) only for the topology of the network, do not consider the exchange of information between nodes; 2) based on the traditional statistical methods, and on the basis of statistics, using data mining method of key nodes, and the change in the flow of network traffic change over time, this method can only static analysis of network key nodes, to analysis of key nodes changes with time and network flow method; 3) focus on the traffic characteristics of the micro field, the whole road network is of little significance. In this paper, we propose a target model based on macroscopic Expressway network. In order to deeply explore the key nodes of freeway, we propose a more complex probabilistic model. This model has high time complexity and can not meet the requirement of real-time application of intelligent transportation system. Based on the small world characteristics of complex networks, this paper introduces the clustering algorithm of expressway network, which is divided into four parts
	
	Therefore, this article from the intelligent transportation according to the practical requirement, aiming at the key nodes of mining lack of research methods, in-depth study of two aspects: the function model between the key nodes and highway traffic conditions put forward a description of expressway, aiming at the limitations of existing key nodes of the complex network, put forward the objective function from the macro level, combined with real-time traffic highway, on key nodes in the expressway real-time position; proposes a classification method of highway network based on community. The resolution limit and extreme for the existing community division method in the degradation characteristics, combined with the characteristics of the highway network, community partition model is established by the highway, to a certain extent to solve the resolution limitation of the traditional method and extreme degradation characteristics.
	
	The main contributions of this paper are as follows:
	
	(1)In this paper, a new model of key nodes in highway network is proposed, which breaks through the limitations of existing complex networks
	
 	(2)In this paper, a new model of community partition based on expressway network is proposed, which can reduce the complexity of the key nodes and make the method practical.
\end{eabstract}

