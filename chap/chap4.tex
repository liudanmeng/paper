% vim:ts=4:sw=4
% Copyright (c) 2014 Casper Ti. Vector
% Public domain.

\chapter{高速公路社群划分方法}

	上一章介绍了面向高速公路的关键节点挖掘模型,并给出了贪心算法。然而根据模型的定义,就算进行简化,认为关键路段已经选出,计算对关键路段进行维护之后的整体网络通行效率也需要$2^n$的时间复杂度。虽然说高速公路上只有部分路段出现过损毁情况,n的规模比较小,勉强可以求解,但是当高速公路网络扩大时,指数级别的复杂度不可接受。

	\section{模型定义}
		模型需要从输入的代表关键节点的离散0-1向量$\bm{y}$,求得高速公路网络通行效率的期望。对于这种输入为整数或整数向量,并且内部具有概率事件的问题,本质上属于随机整数规划问题。在数学优化领域,随机规划是一个涉及不确定性优化问题的框架。比如说两阶段线性规划。决策者在第一阶段采取一些行动,之后发生随机事件影响第一阶段决策的结果。不断调整第一阶段的决策,使得整体期望收益达到最大。

		现有的随机整数规划问题大都是基于班德斯分解方法(Benders Decomposition)进行研究,然而班德斯优化方法要求有两层模型,且两层模型之间互不影响。本研究中,第一层的决策变量$\bm{y}$会直接影响到第二层里面的路网拓扑结构概率,对于这种相互依赖的随机规划问题,现有的研究并没有一些比较合适的优化方法。

		\subsection{高速公路社群划分模型}

			\subsubsection{社群划分可行性分析}

			高速公路网络除具有绝大多数复杂网络的特征外,作为空间网络还具有不同于抽象网络的特性,这些特性决定了高速公路网络的拓扑性质。具体可以归纳为:高速公路交通网络的节点存在于二维地理空间,且有明确的位置;高速公路网络中的边是一种实体联接,具有明确意义,并不是抽象空间中所定义的关系,能够明确表示线路之间的相互关系,线路在整个网络中的重要程度以及网络的局部和全局效率;高速公路交通网络中节点的长程联接需要一定成本,这一特性直接影响着高速公路网络出现小世界行为的可能性;高速公路交通网络中单一节点所能联接的边的数目受到物理空间的限制,这种限制会影响到网络的度分布。

				在以前的高速公路项目研究中,我们发现低跳数的用户占大多数。如图$\ref{fig4}$,可以发现在高速公路中,低跳数的车辆占了大多数,10跳以下的车辆占所有车辆总数的90\%以上。再结合高速公路的异质性,复杂网络的社群性,我们认为高速公路网络应该也具备社群性质,即存在一个个社群,这些社群各自包含一些收费站和高速公路路段,高速公路中的车辆大都从社区内部的节点出发,在同一个社区的另一个节点驶离。社区之间的车辆交流尽量小。

				\begin{figure}[h]
				\centering
						\begin{minipage}{0.8\linewidth}
							\centering
							\includegraphics[width=4.4in]{picture/tiaoshu}
							\caption[这是一个有味道的图]{fig1}
							\label{fig4}
						\end{minipage}%\
				\end{figure}

				为此,抽取某一天的高速公路O-D数据,将有O-D交流的收费站之间连线,流量越多,线的颜色越深,流量越少,线的颜色越浅。如图\ref{fig5},可以较直观的看出高速公路的社群特性。

				\begin{figure}[h]
				\centering
						\begin{minipage}{0.8\linewidth}
							\centering
							\includegraphics[width=4.4in]{picture/shequntexing}
							\caption{fig1}
							\label{fig5}
						\end{minipage}%\
				\end{figure}
			%查重

			\subsubsection{社群划分模型}
				高速公路社群划分的目的是将整个高速公路拓扑结构分成一个个社区,使得社区内部交流尽量多,社区之间的交流尽量少,最终在各自社群分别计算关键节点,分治计算,最后进行合并,达到优化时间复杂度的目的。在此引入基于模块性优化的社区挖掘方法。

				复杂网络具有社群特性,高速公路属于复杂网络的一种。给定高速公路有向图$G=\{V,E\}$,其中V代表收费站(节点)的集合;E表示边的集合。定义社群$c=\{v_1,v_2,...v_m\}$,其中$v_i$是网络中的节点,即收费站或者交叉路口;社群集合$C=\{c_1,c_2,...c_u\}$;其中$v_i \in V$,${c_i}\bigcap {{c_j}}  = \emptyset$,$\sum\limits_{i = 1}^u {\sum\limits_{v \in {c_i}} v  = V}$。

				基于高速公路社群划分的关键节点挖掘算法主要采用分治思想,将一个难以直接解决的大问题,分割成一些规模较小的相同问题,以便各个击破,分而治之。本文主要将路网分成一个个子路网,在子路网中分别计算关键节点,之后再用一定的方法合并。在此需要解决两个问题:

					1)如何分群

					2)分群求解后,如何合并

				传统的复杂网络社群划分系统中,大都是针对虚拟网络(如社交网络)进行研究。高速公路网络和虚拟网络有很大的不同。在虚拟网络中,两个点之间只要有交流,那就代表有边相连;在高速公路中,我们认为只要两个收费站有流量交流,即O-D不为0,那么这两个收费站之间就有边连接(不同于上一章的路网定义)。但是这个边和其他的复杂网络如社交网络不同,社交网络中两个节点之间的空间距离就是1跳,但是对于物理网络来说,两个节点之间的边具有实体距离。高速公路中路段之间的影响也会根据物理距离的变化而变化,这些都是传统方法中没有考虑到的。

				2004年,Newman和Girvan[]提出了一个用于刻画网络社区结构优劣的量化标准,被称作模块化函数。简单的带权模块化函数定义如下:

				\begin{equation}
				Q = \frac{1}{{2m}}\sum\limits_{ij} {[{A_{ij}} - \frac{{{k_i}{k_j}}}{{2m}}]\delta ({c_i},{c_j})}
				\label{eq4}
				\end{equation}

				式\ref{eq4}中,$A_{ij}$表示节点$i$和节点$j$之间的边权;$k_i=\sum\limits_{j} {A_{ij}}$表示所有与节点i相连的边的边权和;$c_i$是指i所属的社群编号;如果$c_i=c_j$,那么$\delta (u,v)=1$,否则等于0;$m=\frac{1}{{2}}\sum\limits_{ij} {A_{ij}}$。

				模块化函数主要用于度量社群划分结构的优劣,现有的基于模块化函数的分群算法都没有考虑高速公路的特性\parencite{},并且在高速公路网络上出现了低分辨率特性和极端退化特性\parencite{}。Newman提出了一种社群挖掘方法\parencite{}:初始的时候没个节点都是一个社群,之后进行迭代,每次迭代时都选择使目标模块函数在Q增加最大的社群进行合并。这个方法虽然在时间效率上很高,但是没有解社群划分的决极端退化特性。Guimera提出一种基于模拟退火的模块性优化方法:初始解是随机生成的社团集合,在每次迭代过程中,采用一定策略,结合当前解生成新的解集,用模块化函数Q判断解集的优劣,最后用模拟退火中的Metropolis准则来决定是否采用该解。这个方法虽然在一定程度上解决了极端退化特性,但是他有一个很严重的问题,在相通的输入集合上,生成的最终结果往往不同,不符合稳定性要求,而且时间复杂度大,求解效率低。Blondel\parencite{}提出快速模块优化方法,他认为首先在局部使用局部模块化函数f获得局部社团,然后再对这些局部社团作为一种超级节点,再进行合并,不断迭代,直到模块化函数Q不再增加为止。这个聚类方法存在聚类社团过大的情况,不符合本文中缩小节点量级,优化算法时间复杂度的目的。针对现有研究的不足,结合高速公路的路网特性,在此提出一种新的面向高速公路的社群划分模型。

				首先定义模块化函数Q:

				\begin{equation}
				\vartriangle Q = [\frac{{\sum_{in} C  + 2{k_{i,in}}}}{{2m}} - {(\frac{{\sum_{tot} C  + {k_i}}}{{2m}})^2}] - [\frac{{\sum_{in} C }}{{2m}} - {(\frac{{\sum_{tot} C }}{{2m}})^2} - {(\frac{{{k_i}}}{{2m}})^2}] - L(i)
				\label{eq5}
				\end{equation}

				公式\ref{eq5}用于判断当节点从一个社区转移到另一个社群的时候,整体路网的社群化结构的变化。根据变化的大小决定节点的社群归属。式中,$\sum_{in} C$表示社群$C$内部的所有边的权重和;$\sum_{tot} C$表示所有与社群$C$中的节点相连的边的权重和;$k_{i,in}$表示$i$到$C$中所有节点之间的连线的权重和;$k_i$表示所有和节点i直接相连的边的权重和;m是路网中所有边的权重之和;$L(i)$是模型罚项,代表i转移社群后,不同社区之间交通流的变化。

				$L(i)$:

				\begin{equation}
				L(i)=\frac{{{k_{i,{c_1}}} - {k_{i,{c_2}}}}}{{{k_{{c_1},{c_2}}}}}
				\label{eq6}
				\end{equation}

				式\ref{eq6}中,${k_{i,{c_1}}}$表示路段i流向社群$c_1$的流量,$k_{i,{c_2}}$代表路段i流向社群$c_2$的流量,${{{k_{{c_1},{c_2}}}}}$表示社群$c_1$,$c_2$中所有节点之间的流量和。

				本文提出的模型中,边的权重不止与两个节点之间的流量有关,还与两个节点之间的物理距离有关。和传统复杂网络不同,节点之间的距离不再由节点之间的最短跳数决定,而是由节点之间的最短物理距离$L$决定:
				\begin{equation}
				L_{ij}=\sum\limits_{e \in E_{ij}} {e}
				\label{eq10}
				\end{equation}

				式\ref{eq10}中,$E_{ij}$是节点i和节点j之间的最短路径中路段的集合。定义边权重:


				\begin{equation}
				W_{ij}=\frac{f_{ij}}{L_{ij}*T}
				\label{eq11}
				\end{equation}

				为了解决传统社群划分中的低分辨率问题,本文中的社群划分方法也采用自底向上的聚类思想,首先定义每一个节点都是一个社群,在每次迭代过程中,利用模块化函数$\vartriangle Q$,依次判断每一个节点所属的社群。同时为了解决社群划分的极端退化问题,我们采用多变权值的思想\parencite{},逐步加大路段距离的权重,减少解集的规模;利用模拟退火思想,定义解集合的数量为$|C|$,利用两次迭代过程中的最终解集合的数量差,结合退火温度,加速模块收敛。分群算法伪代码如下:

				\begin{algorithm}[h]
		        \caption{高速公路社群划分方法}  
		        \label{shequn}
		        \begin{algorithmic}[1] %每行显示行号  
		            \Require 高速车辆O-D数据,高速公路网络拓扑结构,最大社群节点数量
		            \Ensure 高速公路社群划分结果
		            \Function{Community}{$ODMatrix,G={V, E},B$}  
		                \State $res\gets [[\{0,0\},\{1,1\},\cdots,\{n,n\}]]$ 
		                \State $tmp\gets [\{\}]$
		                \State $pre\gets [[\{0,0\},\{1,1\},\cdots,\{n,n\}]]$ 
		                \State $k\gets 0$  
		                \State $l\gets 0$
		                \State $T\gets 100$  
		                \While{$|len(res)-len(pre)| \leqslant T$}
		                	\State $res=res[-1]$
		                	\State $pre=res$
		                	\While{$res[-1] \not\subset res[0:-1]$}
			                	\State $tmp\gets res[-1]$  
			                	\For{$i \in E $}  
			                		\For($C \in tmp \& |C| \le B$)
			                			\If{$\vartriangle Q > k$}  
				                        	\State $l\gets C$  
				                        	\State $k\gets {\vartriangle Q}$  
			                    		\EndIf	
			                		\EndFor
			                    	\State $tmp[l] \gets i$ 
			                	\EndFor
			                	\State $res \ add \ tmp$
		                	\EndWhile
		                	\State T--
		                \EndWhile  
		                \State \Return{$res$}  
		            \EndFunction  
		        \end{algorithmic}  
		    	\end{algorithm} 

		\subsection{基于社群划分的关键节点挖掘模型}
				基于已经分群的高速公路网络,在此提出关键路段挖掘方法。

				分治法的核心是分而治之,首先分割社群,将每个社群看作独立的路网。在每个社群里,用前一章提出的贪心算法选出各个社群中的关键路段,并且计算出每个关键路段被选出后对路段通行效率的增量$\vartriangle L$。忽略不同社团的关键路段之间的相互影响,把分治法的合并问题归类于投资问题:定义资产总额为$B$,每个社团归类于货物$X_i$,$f(y_i*X_{i})$表示当货物$X_i$投资量为$y_i$时,它所带来的收益。在此提出目标模型:

				\begin{align}
				 &Q=Max(\sum\limits_{i = 1}^u {f({y_i}*{X_i})})   \label{fenzhi-merge} \\
				 &Subject \  to. \  \left\{ {\begin{array}{*{20}{c}}
					  {\sum\limits_{i = 1}^u {{y_i}}  \leqslant B} \\ 
					  {{x_i} \geqslant 0} 
				\end{array}} \right.
				\end{align}

				\ref{fenzhi-merge}属于投资问题,可以利用动态规划,在多项式时间内求解。

				动态规划伪代码:

				\begin{algorithm}[h]
		        \caption{关键路段挖掘方法求解}  
		        \label{touzi}
		        \begin{algorithmic}[1] %每行显示行号  
		            \Require 每个社团中选取不同路段的收益,高速公路网络社团结构,最大社群节点数量
		            \Ensure 高速公路关键节点集合
		            \Function{Community}{$CMatrix,C={c_1,c_2,...,c_u},B$}  
		                \State 定义$f_k (x)$:当前k个社团投入x份资源时,最大的通行效率提升量
		                \State $f_0 (x)\gets 0$
		                \State $f_k (0)\gets 0$
		                \State $f_1 (x)\gets CMatrix[1][x]$
		                \State $i=j=1$
		                \While{$i \leqslant u$}
		                	\While{$j\leqslant B$}
		                		\State $f_i (j)=\mathop {{\text{Max}}}\limits_{0 \leqslant y \leqslant j} (CMatrix[i][y]+f_{i-1} (j-y))$
		                		\State $j=j+1$
		                	\EndWhile
		                	\State $i=i+1$
		                \EndWhile  
		                \State \Return{$res$}  
		            \EndFunction  
		        \end{algorithmic}  
		    	\end{algorithm} 

		    	其中,路段收益矩阵CMatrix:

				\[\begin{array}{*{20}{c}}
				  {{c_{11}}}&{{c_{12}}}& \cdots &{{c_{1B}}} \\ 
				  {{c_{21}}}&{{c_{22}}}& \cdots &{{c_{2B}}} \\ 
				   \vdots & \vdots & \ddots & \vdots  \\ 
				  {{c_{u1}}}&{{c_{u2}}}& \cdots &{{c_{uB}}} 
				\end{array}\]
				
				矩阵中,$c_{ij}$表示第i个社群中,选取j条关键路段进行资源投放后,高速公路网络通行效率的提升量。这一数据由贪心算法在每个社群分别求得。

	\section{实验及结果}
		本章节出了针对每一种方法的有效性做出实验,并将基于高速公路社群划分方法的实际效果与通过枚举得到的最优解进行对比。

		基本的社群划分存在分辨率限制和极端退化特性。分辨率限制是指社群划分方法无法发现小于一定规模的社群,极端退化特性是指最终的社群划分结果会收敛于指数数量级的高分解决方案,而不是指向一个或少量最优解。[xxx]采用一种方法解决低分辨率问题:初始化时,将每一个节点看作一个独立的社群,之后根据模块化函数不断循环修正节点的所属社群。这个方法用在高速公路上时,虽然解决了低分辨率社群无法发现的问题,但是最终会产生一系列孤立点(如图\ref{gulidian}),这不符合社群划分的初衷。而且最终结果也没有避开极端退化特性,最终的社群划分结果在一个非常大的解空间中循环。

			\begin{figure}[h]
			\centering
					\begin{minipage}{0.8\linewidth}
						\centering
						\includegraphics[width=4.4in]{picture/liuliangbianquan}
						\caption{fig1}
						\label{gulidian}
					\end{minipage}
			\end{figure}

		图\ref{gulidian}给出了基本的基于模块化函数的分群结果,首先需要指出:使用基础方法的分群结果收敛于一个具有一千多个解的解集合。最终分群结果会在这些解集合内循环。由图我们可以看出两个问题:

		1)存在很多未被分群的孤立点。

		2)很多社群存在物理意义上的交叉收费站。

		孤立点的产生原因有两个,一是这个收费站本身流量较小,与其他站点交流不多;二是这个站点与其他站点之间的交流较为平均,站点不断流动于不同的社团中。图\ref{fenqun2}是基于公式\ref{eq5}的社群划分结果,改图由几百个社群划分解组成的解集中选出,由图可以看出加入物理路经长度的情况下,可以在一定程度上消除孤立点,并且将高速公路划分成较为清晰的几类。但是我们发现仍旧有少量社群,存在物理层面的相互交叉情况,而这种情况不符合高速公路这种物理网络的社群划分特点。

				\begin{figure}
				\begin{minipage}{0.5\linewidth}
					\centering
					\includegraphics[width=2.2in]{picture/liuliangbianquan}
					\caption{fig1}
					\label{fenqun1}
				\end{minipage}%
				\begin{minipage}{0.5\linewidth}
					\centering
					\includegraphics[width=2.2in]{picture/xiaochuguli}
					\caption{fig2}
					\label{fenqun2}
				\end{minipage}
				\end{figure}

		经过数据分析,出现图\ref{fenqun2}中不同社群内部的节点之间存在物理上的交叉情况的原因是——不同社群节点之间的流量差远大于节点之间的距离差。直接将具有交叉节点的社区进行合并虽然简单有效,但是不具有更大规模的适应性,这种方法得到的社群划分效果得不到保证,而且有可能出现过大的社团,不符合社群划分的目的。图\ref{fenqun3}给出了基于模拟退火方法的迭代分群方法结果,该结果最终收敛于由5个结果组成的结果集,基本消除所有孤立点与社群交叉节点。


				\begin{figure}
				\centering
				\includegraphics[width=4.4in]{picture/fenqunjieguo}
				\caption{fig1}
				\label{fenqun3}
				\end{figure}

		根据公式\ref{eq4}给出的模块化函数Q,表\ref{table10}给出了不同社群方法模块化的效果。可以看出将边权与物理距离结合考虑后,模块化效果得到了显著提高;虽然模拟退火方法的时间消耗较大,但是它提供了符合物理网络的分群结果,减少不同社群之间的交叉节点,将不同社群之间节点的相互影响降到最低。

				\begin{table}[h]
				\centering
				\begin{tabular}{|c|c|c|c|}
				\hline
				\hline
				   &   基于流量划分 &   基于流量/距离划分 &   基于变化距离的模拟退火  \\
				\hline
				  模块化效果 &   -1321.21 &   -1025.50 &   -1182.84  \\
				\hline
				  算法效率 &   1min &  30s   &   1.5min  \\
				\hline
				  收敛度 &   $10^3$ &   $10^2$ &   $0-10$  \\
				\hline
				\end{tabular}
				\caption{example of table}
				\label{table10}
				\end{table} 



		图\ref{fenqunend}给出了一天时间内,基于分群算法和简单贪心方法的对比试验;图\ref{end}给出了在一周时间内两种方法的对比试验。和上一章节一样,横坐标表示时间,纵坐标表示路网通行效率的绝对值(路网通行时间)。由图可以看出,简单贪心算法和基于分群算法的关键路段挖掘算法之间的误差较为平稳,并且一直维持在一个较低的水平线上。由图可以看出,分群算法可以在和统计算法相似的时间复杂度上,得到比统计算法优秀的解集。

				\begin{figure}
				\begin{minipage}{0.5\linewidth}
					\centering
					\includegraphics[width=3in]{picture/fenqunend}
					\caption{图片还得再画}
					\label{fenqunend}
				\end{minipage}%
				\begin{minipage}{0.5\linewidth}
					\centering
					\includegraphics[width=3in]{picture/end}
					\caption{图片还得再画}
					\label{end}
				\end{minipage}
				\end{figure}

		下图给出不同方法选出的关键路段集合,图\ref{jihe1}给出了枚举方法选出的关键路段集合,图\ref{jihe2}给出了简单贪心算法给出的关键路段集合,图\ref{jihe3}给出了结合社群划分的关键节点识别算法的结果,图\ref{jihe4}给出了基于统计学的关键节点集合。观察图\ref{jihe1}和图\ref{jihe2},发现两者选取的关键节点具有很强的相似性。

				\begin{figure}
				\begin{minipage}{0.5\linewidth}
					\centering
					\includegraphics[width=3in]{picture/meiju}
					\caption{fig1}
					\label{jihe1}
				\end{minipage}%
				\begin{minipage}{0.5\linewidth}
					\centering
					\includegraphics[width=3in]{picture/tanxin}
					\caption{fig2}
					\label{jihe2}
				\end{minipage}
				\end{figure}

				\begin{figure}
				\begin{minipage}{0.5\linewidth}
					\centering
					\includegraphics[width=3in]{picture/fenqun}
					\caption{fig1}
					\label{jihe3}
				\end{minipage}%
				\begin{minipage}{0.5\linewidth}
					\centering
					\includegraphics[width=3in]{picture/hotsection}
					\caption{fig2}
					\label{jihe4}
				\end{minipage}
				\end{figure}

		误差分析:

				\begin{table}[h]
				\centering
				\begin{tabular}{|c|c|c|}
				\hline
				\hline
				  &  枚举-直接贪心 &  直接贪心-基于社群划分 \\
				\hline
				 一小时 &  14.63\% &  12.89\% \\
				\hline
				 一天 &  13.25\% &  13.26\% \\
				\hline
				 一周 &  13.10\% &  15.61\% \\
				\hline
				 一月 &  12.99\% &  11.59\% \\
				\hline
				\end{tabular}
				\caption{example of table}
				\label{table1}
				\end{table} 

		表\ref{table1}描述了枚举方法和直接贪心方法之间的误差,直接贪心和基于社群划分方法之间的误差。误差由高速路网的通行效率计算,可以看出误差在允许范围内。

		关键节点选取误差分析:

				\begin{table}[h]
				\centering
				\begin{tabular}{|c|c|c|}
				\hline
				\hline
				   &   枚举-直接贪心 &   枚举-基于社群划分 \\
				\hline
				  一小时 &   0.18\% &   0.25\% \\
				\hline
				  一天 &   0.14\% &   0.20\% \\
				\hline
				  一周 &   0.15\% &   0.19\% \\
				\hline
				  一月 &   0.14\% &   0.18\% \\
				\hline
				\end{tabular}
				\caption{example of table}
				\label{table2}
				\end{table} 

		表\ref{table2}分析了关键路段选取情况的误差,采用欧式距离来刻画区别。可以看出,随着数据集的扩大,基于社群划分方法的关键节点准确率逐步上升。

		运行效率分析:

				\begin{table}[h]
				\centering
				\begin{tabular}{|c|c|c|c|c|}
				\hline
				\hline
				   &   枚举 &   直接贪心 &   基于社群划分 &   基于统计 \\
				\hline
				  一小时 &   1day &   30min &   2min &   1min \\
				\hline
				  一天 &   6day &   2h &   5min &   2min \\
				\hline
				  一周 &  7day &   3h &   6min &   5min \\
				\hline
				  一月 &   7day &   3h &   7min &   8min \\
				\hline
				\end{tabular}
				\caption{example of table}
				\label{table3}
				\end{table} 

		由表\ref{table3}可以看出,基于社群划分方法可以将整个算法的时间复杂度再降一个数量级,而结合表\ref{table1}来看,精度误差处于可接受范围$(1/e)$。

	\section{本章小结}
		本章提出了面向高速公路的社群划分方法,首先分析了传统方法的局限性,然后结合高速公路的独有特性,采用多变权值-模拟退火结合的方法,实现符合高速公路网络特点的社群划分方法。