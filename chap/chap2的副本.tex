% vim:ts=4:sw=4
% Copyright (c) 2014 Casper Ti. Vector
% Public domain.

\chapter{复杂网络社群划分相关研究}
	现有的研究主要分布于普通社区挖掘方法和重叠社区挖掘方法

2002年 Girven和Newman引提出社区挖掘的概念。现实世界中的许多复杂系统或以复杂网络的形 式存在、或能被转化成复杂网络。例如:社会系统中 的人际关系网、科学家协作网和流行病传播网,生态 系统中的神经元网、基因调控网和蛋白质交互网,科 技系统中的电话网、因特网和万维网等等。复杂网络 普遍存在着一些基本统计特性,如反映复杂网络具 有短路径长度和高聚类系数之特点的“小世界效 应”;又如表达复杂网络中结点之度服从幂率分布特征的“无标度特性”;再如描述复杂网络中普遍 存在着“同一社区内结点连接紧密、不同社区间结点 连接稀疏”之特点的“社区结构特性”[]。目前,关于 复杂网络基本统计特性的研究已吸引了不同领域的 众多研究者,复杂网络分析已成为最重要的多学科 交叉研究领域之一

随着应用领域的不同,社区结构具有不同的内涵。譬如,社会网中的社区代表了具有某些相近特征的人群、生物网络中的功能组揭示了具有相似功能的生物组织模块、Web网络中的文档类簇包含了大量具有相关主题的web文档、交通网络中的集群区段等等。近10年来,已有很多复杂网络社区挖掘方法被
提出,它们分别采用了来自物理学、数学和计算机科
学等领域的理论和技术,就其依据的原理可分为基
于划分、基于模块性优化、基于标签传播、基于动力
学和基于仿生计算的方法等。2002年,Girvan和Newman
提出了最著名的社区挖掘方法GN(Girvan Newman)。
该算法采用的启发式规则为:社区间链接的边介数
(edge betweenness)应大于社区内链接的边介数,
其中每个链接的边介数被定义为“网络中经过该链接的任意两点间最短路径的条数”。算法GN通过反
复计算边介数,识别社区间链接,删除社区问链接,
以自顶向下的方式建立一棵层次聚类树(dendrogram)。
该算法最大的缺点是计算速度慢。2003年,Tyler等人将统计方法引入算法
GN中,提出一种近似的GN算法。他们的策略是:
采用蒙特卡洛方法估算出部分链接的近似边介数,
而不去计算全部链接的精确边介数。2004年,Radicchi等人提出了用链
接聚类系数(1ink clustering coefficient)取代算法
GN中链接的边介数。他们认为:社区间链接应该很
少出现在短回路(如三角形或四边形)中,否则短回
路中的其他多数链接也会成为社区间链接,从而显
著增加社区间的链接密度。2004年,Newman和Girvan[]¨提出了一个用
于刻画网络社区结构优劣的量化标准,被称之为模
块性函数Q。该算法
中候选解的搜索策略为:选择并合并两个现有的社
区。初始化时,候选解中每个社区仅包含一个结点;
在每次迭代时,算法FN选择使函数Q值增加最大
(或减小最少)的社区对进行合并;当候选解只对应
一个社区时算法结束。通过这种自底向上的层次聚
类过程,算法FN输出一棵层次聚类树(denogram),
然后将对应的函数Q值最大的社区划分作为最终
聚类结果。2005年,Guimera和Amaral[]胡提出了基于模
拟退火的模块性优化算法(simulated annealing,SA)。
该算法首先随机生成一个初始解;在每次迭代中,在
当前解的基础上产生一个新的候选解,由函数Q判
断其优劣,并采用模拟退火策略中的Metropolis准则
决定是否接受该候选解。SA算法产生新候选解的
策略是:将结点移动到其他社区、交换不同社区的结
点、分解社区或合并社区。该算法具有非常好的聚类
质量,但其缺点是运行效率低。2006年,Newman[]朝将谱图理论引入模块性优
化中。2008年,Blondel等人u6】提出了快速模块性优
化方法(fast unfolding algorithm,FUA)。该算法结
合了局部优化与多层次聚类技术。2007年,Raghavan等人。22提出了著名的标签
传播算法(1abel propagation algorithm,LPA).该
算法的流程为:初始化时,为每个结点赋一个唯一标
签;每次迭代中,每个结点采用大多数邻居的标签来更新自身标签;当所有结点的标签都与其多数邻居
的标签相同时,算法结束.2008年,Tib61y等人[]发现标签传播算法
LPA等价于最小化哈密尔敦函数,2009年,Leung等人[]朝将算法LPA作为分析
大规模在线社会网的工具.他们通过研究算法LPA
的优势和限制,讨论了其扩展和优化方面的一些问
题,进而对算法I.PA进行了修正.2009年,Barber等人[]朝将算法LPA等价为一
个优化问题,并给出对应的目标函数20lo年,I。iu等人[26]发现算法LPAm得到的社
区划分具有“每个社区内结点的度之和都相似”的特
性,就是说该算法有陷入局部最优解的倾向.为跳出
局部最优解,他们给出一种多步层次贪婪算法
(muhistep greedy agglomerative algorithm,MSG),每
次可合并多个社区对.进而他们将算法LPAm与
MSG相结合,提出了一个基于模块性优化、层次化
标签传播算法I.PAm+,使标签传播类算法的聚类
性能得到进一步改善.2000年,van Dongen比刊提出了Markov聚类算
法(Markov cluster algorithm,MCI.).该算法主要
是基于Markov动力学理论,通过改变和调节Markov
链呈现出网络社区结构.2007年,杨博等人130j针对符号网络社区挖掘问
题(包括正负权值的网络),提出了基于Markov随
机游走模型的启发式社区挖掘算法(finding and
extracting communities,FEC).2008年,Rosvall等人[]提出了映射平衡算法
infomap.该方法基于最小描述长度(MDI。)原理¨川,
通过信息传播扩散技术探测网络社区结构.2011年,Morfirescu等人[]副研究了一类离散时
间的多agent系统,基于信任度衰减的观点建立动
力学模型.他们将复杂网络视为一个agent网络,其中每个agent拥有一个信念值.2012年,杨博等人[341给出了一个采用Markov
转移矩阵的特征值来评估亚稳态之进出时间的方
法,揭示了网络内在属性与社区结构的数学联系,提
出了分析复杂网络社区结构的谱理论.基于此,定义
了3个刻画社区结构的量,分别为社区之间的分离
度、每个社区的凝聚度和刻画社区结构的谱特征.2007年,Liu等人[]基于每个蚂蚁个体的行为,提
出了一个用于探测邮件社会网社区结构的蚁群聚类
算法.2009
年,Sadi等人¨71采用蚁群优化技术发现网络中的
团,并将这些团视为新结点而构建一个简化网络,然
后通过传统社区挖掘算法来探测社区结构.2010年,刘大有等人[]列从仿生角度出发提
出一个基于Markov随机游走的蚁群算法(ant colony
optimization based on random walk,RWACO).
RWAC()将蚁群算法框架作为基本框架.以Markov
随机游走模型作为启发式规则,通过集成学习的思
想将蚂蚁的局部解融合为全局解,并用其更新信息素矩阵.通过“强化社区内连接,弱化社区间连接”这
一进化策略逐渐呈现出网络的社区结构.
