% vim:ts=4:sw=4
% Copyright (c) 2014 Casper Ti. Vector
% Public domain.

\specialchap{总结与展望}
% 中文测试文字。
	交通系统中的关键节点识别非常重要。在高速公路系统中,关键节点的损毁会对整个系统的性能造成显著的影响,带来重大的经济损失。所以识别关键节点,在发生事故或者自然灾害之前进行维护巩固,在发生事故后进行快速修复,维护网络完整性非常重要。

	\subsection{主要工作}
		本文的主要工作和创新点如下:

		1)结合高速公路特性,提出高速公路关键路段挖掘模型,证明模型的贪心可解性,给出贪心解法。

			重要节点一般是指那些影响可以快速地波及到网络中大部分节点的节点。 比如说,在对一个无标度网络的攻击中,针对少量重要节点进行攻击,整个网络会很快瓦解;网络的“小世界特性”和“无标度特性”的发现,加深了网络科学的研究热度。 早期网络科学研究的热点是发现跨越不同网络的宏观上的普适规律,而现在早已转变为从微观的节点链路层面以及中观的社团结构和群组结构角度去研究不同网络的特征。重要节点的挖掘研究也逐渐转为以微观研究为主。然而,从管理者的角度来说,过于微观的研究又无法体现高速公路系统的宏观特性。所以在此本文提出一种结合宏观(目标函数基于宏观理念)微观(基于路段损毁的随机规划问题)的高速公路关键路段挖掘模型,有效的解决了微观研究无法很好的顾及整体的问题。

		2)结合高速公路的复杂网络,提出一种基于复杂网络社群划分的关键路段挖掘方法。

			传统的复杂网络社群划分与高速公路不同,首先高速公路是一种相当稀疏的复杂网络,网络中的拓扑结构特性不是很复杂;其次高速公路和普通的复杂网络不同,他的不同的节点之间具有物理空间距离,和其他复杂网络如社交网络中的距离概念不同。所以传统的复杂网络社群划分方法已经不再适用。在此引入可变权值方法,有效的解决了传统分群算法中的低分辨率特性以及极端退化特性。同时采用模拟退火思想加速模型收敛过程。
	\subsection{未来工作展望}

		本文工作虽然具有一定的创新性和实用性,但仍然存在一些局限和不足,需要在今后的研究中进一步探讨和完善,主要包括以下几个方面:

		1)由于时间限制,目前只在安徽路网上做完了原型系统。下一步的工作是将系统复用到全国的高速公路网络。

		2)目前分群算法结合高速公路的物理空间特性,在社群划分效果上可能有一定的损失。随着中国交通建设的不断完善,高速公路监测数据也越来越丰富。希望之后可以结合新的高速数据集,进一步完善高速公路关键路段挖掘方法。

