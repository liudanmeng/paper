% vim:ts=4:sw=4
% Copyright (c) 2014 Casper Ti. Vector
% Public domain.

\specialchap{总结与展望}
% 中文测试文字。
	交通系统中的关键节点识别非常重要。在高速公路系统中,关键节点的损毁会对整个系统的性能造成显著的影响,带来重大的经济损失。所以识别关键节点,在发生事故或者自然灾害之前进行维护巩固,在发生事故后进行快速修复,维护网络完整性非常重要。

	\subsection{主要工作}
		本文的主要工作和创新点如下:

		1)针对高速公路的实际特性,用概率来模拟高速公路的损毁情况,提出一种判定高速公路路段重要程度的研究模型。

			重要节点一般数量非常少, 但其影响却可以快速地波及到网络中大部分节点. 例如, 在对一个无标度网络的蓄意攻击中, 少量最重要节点被攻击就会导致整个网络瓦解;网络的“小世界特性”和“无标度特性”的发现掀起了网络科学持续 10 多年至今丝毫没有降温的研究热潮. 网络科学研究的热点逐渐从早期发现跨越 不同网络的宏观上的普适规律转变为着眼于从中观 (社团结构、群组结构)和微观层面(节点、链路)去解释不同网络所具有的不同特征。重要节点的挖掘研究也逐渐转为以微观研究为主。然而,从管理者的角度来说,过于微观的研究又无法体现高速公路系统的宏观特性。所以在此本文提出一种结合宏观(目标函数基于宏观理念)微观(基于路段损毁的随机规划问题)的高速公路关键路段挖掘模型,有效的解决了微观研究无法很好的顾及整体的问题。

		2)针对高速公路的空间物理特性,提出一种结合物理网络特性的复杂网络社群划分方法。

			传统的复杂网络社群划分与高速公路不同,首先高速公路是一种相当稀疏的复杂网络,网络中的拓扑结构特性不是很复杂;其次高速公路和普通的复杂网络不同,他的不同的节点之间具有物理空间距离,和其他复杂网络如社交网络中的距离概念不同。所以传统的复杂网络社群划分方法已经不再适用。在此引入可变权值方法,有效的解决了传统分群算法中的低分辨率特性以及极端退化特性。同时采用模拟退火思想加速模型收敛过程。
	\subsection{未来工作展望}

		本文工作虽然具有一定的创新性和实用性,但仍然存在一些局限和不足,需要在今后的研究中进一步探讨和完善,主要包括以下几个方面:

		1)由于时间限制,目前只在安徽路网上做完了原型系统。实际上安徽一个省的路网并不大,文中的优化方法也都是针对大规模路网提出的。所以下一步的工作就是将整个系统复用到全国的高速公路网络中去。

		2)目前分群算法尽量结合高速公路的物理空间特性,在分群效果上有一定的损失。现有的研究数据都是基于2010到2012年的历史数据进行研究,随着中国交通建设的不断完善,人们生活水平的不断提高,相信在高速公路上选择短途旅行的游客会越来越多。希望之后可以在最新的数据上进行研究,完善高速公路社群挖掘方法。

