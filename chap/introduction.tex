% vim:ts=4:sw=4
% Copyright (c) 2014 Casper Ti. Vector
% Public domain.

\specialchap{引言}
% 中文测试文字。
\section{研究背景}
\subsection{高速公路交通研究背景}

交通系统是人类活动不可缺少的一部分。据估计,每天平均有40\%的人口在路上花费至少1小时。近几年来,人们变得越来越依赖于交通系统,对于交通系统管理人员来说,机遇和挑战共存。首先,交通拥堵已成为一个日益严重的问题。全球范围内的道路上的车辆增加,根据调查,截止至2016年初,北京共有544万辆车,比2014年初增加了50万辆。这些激增的车辆会对道路系统产生严重的压力,极大的增加拥堵以及拥堵后的损耗。拥堵会导致燃油消耗增加,空气污染,以及实施公共交通计划的困难。车流流量过多时,交通事故风险与交通运输系统中的膨胀增加,交通事故之后的恢复时间与恢复代价也会急剧增加。在中国,2009年的交通事故死亡人数约有7万人,在2015年达到9万人。美国联邦公路管理局公布的报告显示,发生在城市的交通事故约占所有拥堵延误的50\% - 60\%。毫无疑问,如何高效处理交通事故和预测事故发生点,一旦世故发生,最大限度地减少其影响是一个核心问题。第三,资源相对有限,尤其是中国高速公路正在逐渐走向免费,因此很难全面建立新的基础设施。同时,运输系统的有效性也越来越依赖于一个国家的处理紧急情况的能力(例如,大规模疏散和安全增强)。一个国家的技术竞争力,其经济实力和生产能力,在很大程度上取决于其交通系统性能。

			在过去的二十年中,智能交通系统(ITS)已成为一种提高交通系统的性能,提高行车安全有效的方式,并且为旅客提供更多的选择。上述一些问题可以通过智能交通系统,实施新的交通政策来解决。这些政策是基于高速公路的特性提出的,高速公路的特性主要有网络性,小世界特性,社区结构特性、动态性以及周期性。

\subsubsection{网络性}
    纵横交错的道路构成了复杂的交通路网,这使得交通系统具有了网络性质。网络中,不同的收费站构成了节点,相邻收费站之间的道路构成了网络中的边,节点之间通过车辆来交流。高速公路网络化,使得交通系统中的车流行为更加复杂,这对交通研究方法提出了更高的要求。网络化所引发的复杂性在于,路网中不同空间位置的交通流行为并非孤立产生,而是相互间存在着紧密关系,路网愈加庞大,关系愈为复杂。两个路网中不同空间位置的交通流之间存在着紧密关联,例如较多的车流从某些特定的入口进入路网,又从某些特定的出口流出路网,并且不同出口共享着某些车流来源。然而,传统的以单位置点为研究对象的交通流分析方法并不能有效利用车流之间的关联信息,因此它们已经无法再适用于网络化的交通系统。交通流之间的关联性促生了从路网视角进行全局交通流分析的需求,要求将路网中多节点的交通流行为同时进行学习。

\subsubsection{小世界特性}

    小世界特性(Small world theory)又被称之为是六度空间理论或者是六度分割理论(Six degrees of separation)。小世界特性指出:社交网络中的任何一个成员和任何一个陌生人之间所间隔的人不会超过六个。在高速公路网络中,小世界特性的表现有所不同:网络中绝大部分车辆的跳数(车辆旅行途中经过的道路数量)小于6个.
    
\subsubsection{无标度特性}
				现实世界的网络大部分都不是随机网络,少数的节点往往拥有大量的连接,而大部分节点却很少,节点的度数分布符合幂率分布,而这就被称为是网络的无标度特性(Scale-free)。将度分布符合幂律分布的复杂网络称为无标度网络。在高速公路网络中,统计发现少量节点占有着大多数车辆。(上图)
\subsubsection{社区结构特性}
				人以类聚,物以群分。复杂网络中的节点往往也呈现出集群特性。例如,社会网络中总是存在熟人圈或朋友圈,其中每个成员都认识其他成员。集群程度的意义是网络集团化的程度;这是一种网络的内聚倾向。连通集团概念反映的是一个大网络中各集聚的小网络分布和相互联系的状况。在高速公路网络中,这个特性体现在:高速公路的节点组成一个个社团,这些社团绝大部分车辆都驶向社团内部。
\subsubsection{动态性以及周期性}
				交通路网是一种动态系统,随着时间的变化,其内部的交通流规律与运行模式都在不断变化。交通现象具有周期性,典型的例子是以日为周期的交通流交替运行模式。

\subsection{关键节点挖掘研究背景}

复杂网络的重要节点是指相比网络其他节点而言, 能够在更大程度上影响网络的结构与关键词功能的一些特殊节点. 近年来, 节点重要性排序研究受到越来越广泛的关注, 不仅因为其重大的理论研究意义, 更因为其广泛的实际应用价值. 由于应用领域极广, 且不同类型的网络中节 点的重要性评价方法各有侧重, 学者们从不同的实际问题出发设计出各种各样的方法.几乎所有的复杂系统(比如社会、生物、信息、技术、 交通运输系统)都可以自然地表示为网络. 其中, 节 点代表系统的各种构成要素, 节点间的连边表示要 素之间的联系. 最复杂的人类社会系统就可以用一 个社会网络刻画, 节点是人, 人与人之间的各种关系 构成社会网络中的链接.本文最核心的研究问题就是如何识别这些重要的节点. 所谓的重要节点是指相比网络其他节点而言能够在更大程度上影响网络的结构与功能 的一些特殊节点.所以传统的研究方向也主要是在这两个方向上进行。

			在传统的研究中,网络结构是指整个网络的成网性,包括度空间分布、平均距离、连通性、聚类系数、度相关性等。 网络功能涉 及网络的抗毁性、传播、同步、控制等。

			重要节点一般数量非常少, 但其影响却可以快速地波及到网 络中大部分节点. 例如, 在对一个无标度网络的蓄 意攻击中, 少量最重要节点被攻击就会导致整个网 络瓦解; 网络的“小世界特性”和“无标度特性”的发现掀起了网络科学持续 10 多年至今丝毫没有降温的研 究热潮. 网络科学研究的热点逐渐从早期发现跨越 不同网络的宏观上的普适规律转变为着眼于从中观 (社团结构、群组结构)和微观层面(节点、链路)去解 释不同网络所具有的不同特征。这一转变, 是因为 随着研究的深入, 人们发现宏观指标不能很好表现 网络结构和功能上的特征, 真正精细可靠的解释, 哪 怕是针对宏观现象, 也必须立足于微观上的深入认 识. 类似地, 多年以前, 一批学者就提倡关注网络结 构和功能的相互影响[10], 但是早期的研究都集中在 网络宏观或者中观上的一些特征与网络具体功能表 现之间的关系, 所得到的一些结论, 譬如“热力学极 限下无标度网络传染病(SIS 模型)没有阈值, 随机网 络有阈值”[11]等, 都只是一些统计上有意义, 大多数 情况下正确, 定性上可以部分解释, 定量上无法开展 预测的结果. 这是因为宏观指标以及基于宏观量的 运算, 已经把很多个体的特征进行了“平均化”, 而一 些非常关键个体的表现被这种“平均化”淹没了. 还 是回到刚才的例子, 如果我们从个体出发, 比如仔细 考虑一个节点在 SIS 传播动力学中可能的自维持特 性, 就会得到颠覆性的结论: “热力学极限下随机网 络上的 SIS 模型也没有阈值”[12]. 可见, 基于微观层 面, 即节点个体的分析, 有望揭示网络功能上精细入 微的特征. 总之, 随着网络科学研究从整体宏观到个 体微观的转变, 重要节点的排序和挖掘已成为近年 来的研究热点。

			图xxx给出了复杂网络在传统方法中的样例。

			上述复杂网络的关键节点方法经过不断的研究与发展,已经成功应用于众多研究领域中去。然而这些方法都有一定的局限性,或者是局限于微观节点,忽略了宏观节点之间的关系;或者是只关注路网的拓扑结构,没有关注路网的内在车流信息。
			

\subsection{社群划分研究背景}
现实世界中的许多复杂系统或以复杂网络的形 式存在、或能被转化成复杂网络.例如:社会系统中 的人际关系网、科学家协作网和流行病传播网,生态 系统中的神经元网、基因调控网和蛋白质交互网,科 技系统中的电话网、因特网和万维网等等.复杂网络 普遍存在着一些基本统计特性,如反映复杂网络具 有短路径长度和高聚类系数之特点的“小世界效 应”;又如表达复杂网络中结点之度服从幂率分布特征的“无标度特性”[]一;再如描述复杂网络中普遍 存在着“同一社区内结点连接紧密、不同社区间结点 连接稀疏”之特点的“社区结构特性”[].目前,关于 复杂网络基本统计特性的研究已吸引了不同领域的 众多研究者,复杂网络分析已成为最重要的多学科 交叉研究领域之一[].图[]中给出了上述统计特性 的直观描述.


			传统的研究方法主要分1)基于划分的社区挖掘方法,即先找出社区间的所有链接,接着将它们全部删除,最后每个连通分支对应着一个社区;2)基于模块性优化的社区挖掘方法,即提出了一个用于刻画网络社区结构优劣的量化标准,被称之为模块函数Q;3)基于标签传播的社区挖掘方法,即它没有特定的目标函数,而是通过一种直觉、富有启发的思想推断社区结构和设计算法.标签传播类方法的启发式规则为“在具有社区结构的 网络中,任一结点都应当与其大多数邻居在同一个 社区内”.4)基于动力学的社群划分方法,比如说基于Markov随机游走理论的启发式求解策略;5)基于仿生计算的社区挖掘方法,主要包括蚁群算法和遗传算法。

			上述研究或者基于没有实体的非物理复杂网络,如社交网络,或者具有相应的缺陷,如基于模块性优化方法中的分辨率限制与极端退化特性。

\section{研究内容}
    综上所述,本文从交通实际问题的角度出发,针对现有复杂网络关键节点挖掘技术的不足,深入开展下述两项研究内容:
    
		(1)提出一种度量高速公路节点重要性的研究模型,可以在宏观层面反映这些节点对高速公路网络的影响高,挖掘高速公路关键节点
		
		(2)提出一种结合高速公路网络特性的社群划分算法,达到较强的收敛性与低误差。
		
\section{论文结构}
    第一章为绪论,介绍了本文的研究背景,提出了本文的研究内容。第二章
介绍了复杂网络关键节点研究的相关工作,结合交通问题的特点分析了现有方法的优势与不
足。第三章对复杂网络社群划分方法及其相关研究进行了介绍,通过对现有社群划分方法的分类对比,分析了它们的优势与不足。从第四章开始的后续章节将论述本
文的主要研究内容。第四章提出了一种复杂网络关键节点挖掘模型,给出
了详尽的理论分析,并在多个数据集下进行了验证。第五章
提出了一种基于高速公路交通网络的社群划分模型,给出了高效的优化算法和详
尽的理论分析,并在多个数据集下的进行了验证。第六章给出了混合模型在真实交通场景下的应用
实例。第七章给出了全文的总结与未来工作展望。