% vim:ts=4:sw=4
% Copyright (c) 2014 Casper Ti. Vector
% Public domain.

\chapter{复杂网络关键节点挖掘相关研究}

复杂网络的重要节点是指相比网络其他节点而言, 能够在更大程度上影响网络的结构与关键词功能的一些特殊节点. 近年来, 节点重要性排序研究受到越来越广泛的关注, 不仅因为其重大的理论研究意义, 更因为其广泛的实际应用价值. 由于应用领域极广, 且不同类型的网络中节 点的重要性评价方法各有侧重, 学者们从不同的实际问题出发设计出各种各样的方法.几乎所有的复杂系统(比如社会、生物、信息、技术、 交通运输系统)都可以自然地表示为网络. 其中, 节 点代表系统的各种构成要素, 节点间的连边表示要 素之间的联系. 最复杂的人类社会系统就可以用一 个社会网络刻画, 节点是人, 人与人之间的各种关系 构成社会网络中的链接.本文最核心的研究问题就是如何识别这些重要的节点. 所谓的重要节点是指相比网络其他节点而言能够在更大程度上影响网络的结构与功能 的一些特殊节点.所以传统的研究方向也主要是在这两个方向上进行。

			在传统的研究中,网络结构是指整个网络的成网性,包括度空间分布、平均距离、连通性、聚类系数、度相关性等。 网络功能涉 及网络的抗毁性、传播、同步、控制等。
			
\section{基于节点临近}
该方法是最简单直观的方法, 度中心性考察 节点的直接邻居数目, 半局部中心性考虑了节点4层 邻居的信息. k-壳分解可以看作度中心性的一种扩展, 它根据节点在网络中的位置来定义其重要性, 认为 越是在核心的节点越重要.
\subsection{度中心性}
社会网络分析中, 节点的重要性也称为“中心 性”, 其主要观点是节点的重要性等价于该节点与其他 节点的连接使其具有的显著性[]. 度中心性(degree centrality)[]认为一个节点的邻居数目越多, 影响力 就越大, 这是网络中刻画节点重要性最简单的指标.节点 vi 的度, 记为 ki, 是指与 vi 直接相连的节点的数 目, 是节点最基本的静态特征. 在有向网络中, 根据 连边的方向不同, 节点的度有入度和出度之分. 在含 权网络中节点度又称为节点的强度(strength), 定义 为与节点相连的边的权重之和. 度中心性刻画的是 节点的直接影响力[], 它认为一个节点的度越大, 能直接影响的邻居就越多, 也就越重要. 值得注意的 是, 不同规模的网络中有相同度值的节点有不同的 影响力, 为了进行比较, 定义节点 vi 的归一化度中心 性指标为:

			加入公式

			其中, ki iaij , aij即网络邻接矩阵A中第i行第j列元素, n 为网络的节点数目, 分母 n1 为节点可能 的最大度值. 在有向网络中入度和出度有不同的意 义(如社交网络中入度代表受欢迎程度, 出度代表合 群程度), 一般会分别计算入度和出度的中心性.

			度中心性指标拥有简单、直观、计算复杂度低等 特点. 在网络鲁棒性和脆弱性研究中, 针对无标度网 络或指数网络, 如果攻击前一次性选择若干个攻击 目标, 采用度中心性指标的攻击效果比介数中心性、 接近中心性、特征向量中心性要好(参见 6.1 节). 度 中心性指标的缺点是仅考虑了节点的最局部的信息, 是对节点最直接影响力的描述, 没有对节点周围的 环境(例如节点所处的网络位置、更高阶邻居等)进行 更深入细致地探讨, 因而在很多情况下不够精确.
\subsection{半局部中心性}
度中心性指标计算方便简单, 但实际效果欠佳. 基于全局信息的方法, 如在下一节中介绍的介数中 心性和接近中心性指标, 虽然具有较好的刻画节点 重要性的能力, 但计算复杂度太高, 难以在大规模网 络上使用. 为了权衡算法的效率和效果, Chen 等人[] 提出了一种基于半局部信息的节点重要性排序方法, 简称半局部中心性(semi-local centrality). 首先定义 N(w)为节点 vw 的两层邻居度,其值等于从 vw 出发 2 步内可到达的邻居的数目, 然后定义

			插入公式

			其中(j)表示节点 vj 的一阶邻居节点的集合. 最终 节点 vi 的局部中心性定义为

			插入公式

			可见, 半局部中心性涉及了节点的四阶邻居信息. 文献[]用 D-S 证据理论(参见 5.6 节)将本方法推广到了含权网络. 文献[]指出半局部中心性方法的计算复杂度随网络规模线性增长, 消耗非常少的计算时间, 就能够得到远好于度中心性和介数中心性的排序结果. 近期, Chen 等人[]还提出一种针对有向网络的半局部算法(ClusterRank), 该算法不仅考虑了邻居节点的数量, 还考虑了聚类系数对信息传播的影响: 聚类系数越大越不利于信息的广泛传播.两个数据集上的实验结果显示 Cluster- Rank算法优于PageRank和LeaderRank算法, 并且计 算复杂度最低.
\subsection{k-壳分解法}
度中心性仅考察节点最近邻居的数量, 认为度 相同则重要性相同. 然而, 近期的一些研究表明在刻 画节点重要性的时候节点在网络中的位置也是至关 重要的因素. 在网络中, 如果一个节点处于网络的核 心位置, 即使度较小, 往往也有较高影响力; 而处在 边缘的大度节点影响力往往有限. 基于此, Kitsak 等 人[]提出用 k-壳分解法(k-shell decomposition)确定网 络中节点的位置, 将外围的节点层层剥去, 处于内层 的节点拥有较高的影响力. 这一方法可看成是一种 基于节点度的粗粒化排序方法. 具体分解过程如 下[]: 网络中如果存在度为1的节点, 从度中心性的 角度看它们就是最不重要的节点. 如果把这些度为 1 的节点及其所连接的边都去掉, 剩下的网络中会新 出现一些度为 1 的节点, 再将这些度为 1 的节点去掉, 循环操作, 直到所剩的网络中没有度为 1 的节点为止. 此时, 所有被去掉的节点组成一个层, 称为 1-壳(记 为 ks=1). 对一个节点来说, 剥掉一层之后在剩下的 网络中节点的度就叫该节点的剩余度. 按上述方法 继续剥壳, 去掉网络中剩余度为 2 的节点, 重复这些 操作, 直到网络中没有节点为止. 更广泛地, 可定义 初始度为 0 的孤立节点属于 0-壳, 即 ks=0. 网络中的 每一个节点属于唯一的一层, 显然所有节点均满足k  ks .
			图1给出一个k-壳分解的示例. 其中(a)为原网络,(b), (c), (d)分别表示 1-壳, 2-壳和 3-壳. 可见, 大度节点有可能因处于核心位置而拥有较大的 ks 值(如图 1(d)中的深色节点), 也可能因为处于边缘而具有较 小的 ks 值(如图 1(b)中的深色节点). 在这个方法下, 大度节点不一定是重要节点.

			插入图

			k-壳分解法计算复杂度低, 在分析大规模网络的层级结构等方面有很多应用. 然而, 此方法也有一定局限性. 第一, k-壳分解法有很多不能发挥作用的场景. 比如在树形图, 规则网络和 BA 网络[]中, 所有(或大部分)节点都会被划分在同一层. 更极端的例子是星形图, 显然中心节点有最强的传播能力, 但是 k-壳分解的时候, 星形网络的所有节点会被划分在同一层(ks=1). 第二, k-壳分解法的排序结果太过粗粒化,使得节点的区分度不大. k-壳分解法划分的层级比度中心性方法划分的层级少很多, 很多节点处在同一层上, 它们之间的重要性难以比较. 第三, k-壳分解法在网络分解时仅考虑剩余度的影响, 这相当于认为同一层的节点在外层都有相同的邻居数目, 显然不合理. Zeng 等人[]提出了在每一步剥去一部分外围节点之后, 同时考虑节点剩余的邻居数kr 和节点 i已经移除的邻居数 k e 的方法, 定义节点 v 的混合度 ii为 kr ke , 根据新的混合度值对网络继续分层. ii这种采用混合度值的 k-壳分解法能够很好地区分树 形图以及 BA 网络中不同节点的传播能力, 并且分层 的层数大大增加(甚至可超过度中心性), 提高了节点 传播能力的区分度. 另外, Liu 等人[]指出壳数相同 的节点传播能力差距可能很大, 并提出了一种可以 进一步区分具有相同壳数的节点的传播能力的排序方法, 从而较 Kitsak 等人[]的方法有所进步; Hu 等 人[]将 k-壳分解法与社区结构相结合, 提出一种改 良指标, 在SIR模型上的实验表明该方法较Kitsak等 人的方法略佳.
			
			
			
\section{基于路径临近}
在交通、通信、社交等网络中存在着一些度很小 但是很重要的节点, 这些节点是连接几个区域的“桥 节点”, 它们在交通流和信息包的传递中担任重要的 角色. 此时, 刻画节点重要性就需要考察网络中节点 对信息流的控制力, 这种控制力往往与网络中的路 径密切相关. 基于最短路径的排序方法假设网络中 的信息流只经过最短路径传输, 而真实的通信网络 中必须考虑负载平衡, 容错机制, 服务水平协议 (SLA)等[]. 除了路径长度, 路径上的中间节点个数 对传播也有不可忽视的影响. 一对节点的中间节点 会增加这两个节点之间进行互动所需要的消耗. 第 一, 中间节点越多, 一对节点之间互动所需要的时间 就越长; 第二, 中间节点相当于在一对进行互动的节 点之间引入了“第三方”, 这会使传递的信息失真或 者延迟传递. 另一方面, 从提高网络的可靠性和抗毁 性角度看, 任意节点对之间的路径数目越多, 网络的 鲁棒性就越高. 此外类似于“桥节点”, 程学旗等人提 出了刻画网络边重要性的指标用来寻找“桥链路”, 相关讨论参见文献[].

\subsection{离心中心性}
在连通网络中, 定义 dij 为节点 vi 与 vj 之间的最 短路径长度, 也称最短距离, 一个节点 vi 的离心中心 性 (Eccentricity)为它与网络中所有节点的距离之中 的最大值[], 即:

			插入公式

			网络直径定义为网络 G 中所有节点的离心中心 性中的最大值, 网络半径定义为所有节点的离心中 心性值中的最小值. 显然, 网络的中心节点就是离心 中心性值等于网络半径的节点, 一个节点的离心中 心性与网络半径越接近就越中心. 要强调的是, 网络 直径在复杂网络研究中还有多种不同的定义, 例如 Albert 等人[]在研究万维网的时候定义网络直径为 网络中所有节点对的最短路径的平均值. 离心中心 性的缺点是极易受特殊值的影响, 如果一个节点与大部分节点的距离都很小, 只与极小部分节点的距 离很大,这个节点的离心中心性仍然会取其中的最大值.接近中心性则采取距离平均值的方式克服了 这一缺点.
\subsection{接近中心性}
接近中心性(closeness centrality) 通过计算节
点与网络中其他所有节点的距离的平均值来消除特 殊值的干扰. 一个节点与网络中其他节点的平均距 离越小, 该节点的接近中心性就越大. 接近中心性也可以理解为利用信息在网络中的平均传播时长来确
定节点的重要性. 平均来说, 接近中心性最大的节点 对于信息的流动具有最佳的观察视野. 对于有 n 个节 点的连通网络, 可以计算任意一个节点 vi 到网络中 其他节点的平均最短距离:

			插入公式

			di 越小意味着节点 vi 更接近网络中的其他节点, 于是 把 di 的倒数定义为节点 vi 的接近中心性, 即:

			插入公式

			上面定义的缺点是仅能用于连通的网络中, 文献[] 在研究网络效率时对上式进行了改进, 使其能够用 于非连通网络中, 即:

			插入公式

			如果节点 vi 和 vj 之间没有路径可达则定义dij , 即1dij 0. 接近中心性利用所有节点对之 间的相对距离确定节点的中心性, 在研究中应用非 常广泛, 但时间复杂度比较高.
\subsection{Katz中心性}
与接近中心性不同, Katz 中心性不仅考虑节点对 之间的最短路径, 还考虑它们之间的其他非最短路 径[]. Katz 中心性认为短路径比长路径更加重要, 它 通过一个与路径长度相关的因子对不同长度的路径 加权. 一个与 vi 相距有 p 步长的节点, 对 vi 的中心性的贡献为sp(s(0,1)为一个固定参数). 设lp为从 ij节点 vi 到 vj 经过长度为 p 的路径的数目. 显然 A2 l 2 a a , 其中元素l 2即从节点 v 到ij k ik kj ij ivj 经过的边数为 2 的路径的数目, 同理我们可以得到A3, A4···Ap···, 将这些值赋予不同权重然后相加, 便可 以得到一个描述网络中任意节点对之间路径关系的 矩阵:

			插入公式

			其中,I为单位矩阵.K矩阵中第i行j列对应的元素kij 实际上就是我们所熟知的节点 vi 和 vj 的 Katz 相似性[]. 为保证 K 可写成公式(8)右侧的矩阵形式, 要求参数 s 小于邻接矩阵的最大特征值的倒数. 由此可定义一个节点 vj 的 Katz 中心性为矩阵 K 第 j 列元素的和:

			插入公式

			Katz 中心性使用矩阵求逆的方法虽然比直接数 路径数目简单, 但时间复杂度依然比较高. 另一方面, 在考虑所有路径长度时, 如果节点 vi 与 vj 之间存在长 度为 p 的路径, 在使用 K 矩阵计算节点间长度为 p 的 奇数倍的路径时, 这条路径会被重复计算多次. 衰减 因子 s 的引入正好削弱了这些由于重复计算产生的 对中心性值的影响, 特别是当 s 很小时, 高阶路径的 贡献就非常小了, 使 Katz 指标的排序结果接近于局 部路径指标. Katz 中心性主要用在规模不太大, 环路 比较少的网络中. 受到 Katz 中心性指标的启发, 我 们还可以应用其他刻画节点间相似性的指标[]来定 义节点中心性.
\subsection{信息指标}
信息指标(information indices)[]通过路径中传播的信息量来衡量节点重要性. 该方法假定信息在一条边上传递的时候存在一定的噪音, 路径越长噪音就越大. 一条路径上的信息传输量等于该路径长度的倒数. 一对节点(vi, vj)间能够传输的信息总量就等于它们之间所有路径传输的信息量之和, 记为 qij.值得注意的是, 如果我们把网络看成一个电阻网络,每条边的电阻记为 1, 则 1/qij 相当于以 2 个节点 vi 和vj 为两端点的电阻值(qij 相当于电导)[], 于是我们可以通过计算矩阵R(r)(DAF)1获得q, 其中 ij ijD 是 n 阶对角矩阵, 对角线元素都是对应节点的度值, 非对角线元素为 0, F 是每个元素均为 1 的 n 阶方阵. 由此可得该网络中每一对节点(vi, vj)间通过所有路径 能够传播的信息总量为

			插入公式

			最后, 用调和平均数的方法定义节点 vi 的中心性指标(有时也采用算术平均数)[]:

			插入公式

			信息指标考虑了所有路径, 并可通过电阻网络 简化繁复的计算过程. 该方法可以很容易地扩展到 含权网络, 也适用于非连通的网络.
			可见, 无论是接近中心性、Katz 中心性还是信息 指标, 它们的思路是一致的. 如果用一个矩阵 M=(mij) 来表示网络中所有节点之间的关系, M 的每一个元素 mij 刻画了节点 vi 和 vj 之间的某种联系, 这个联系既 可以是它们之间的距离(如接近中心性), 也可以是某 种相似性, 于是一个节点 vi 的重要性可表示为Centrality(i)   j m . 由此可见, 只要我们能够给出 ij一种刻画节点关系的方式, 就能够基于这个方法定 义一个节点的中心性.
\subsection{介数中心性}
通常提到的介数中心性(betweenness centrality) 一般指最短路径介数中心性(shortest path BC), 它认 为网络中所有节点对的最短路径中(一般情况下一对 节点之间存在多条最短路径), 经过一个节点的最短 路径数越多, 这个节点就越重要. 介数中心性刻画了 节点对网络中沿最短路径传输的网络流的控制力. 节点vi 的介数定义为

			公式

			其中, g 为从节点 v 到 v 的所有最短路径的数目, gstst st为从节点 vs 到 vt 的 gst 条最短路径中经过 vi 的最短路 径的数目. 显然, 当一个节点不在任何一条最短路径 上时, 这个节点的介数中心性为 0, 比如星形图的外 围节点. 对于一个包含 n 个节点的连通网络, 节点度 的最大可能值为 n1, 节点介数的最大可能值是星形 网络中心节点的介数值: 因为所有其他节点对之间 的最短路径是唯一的并且都会经过该中心节点, 所 以该节点的介数就是这些最短路径的数目, 于是得到一个归一化的介数:

			介数中心性可用于设计网络的通信协议、优化网 络部署、检测网络瓶颈等. 王延庆[]将介数应用于负 载网络, 提出用过载函数法研究网络的连接失效问 题. 此外, Goh 等人[]提出的负载中心性(traffic load centrality)采用类似网络中信息包传递的机制: 每一 对节点之间沿着最短路径传输一个单位的网络流, 如果最短路径不止一条, 则在几条最短路径的分叉 处将网络流平均分配到这些最短路径上. 忽略时延, 网络中所有节点对之间都互不干扰地传输一个单位 的信息流时, 一个节点上传输过的网络流的数量称 为该节点的负载. 一个节点的负载越大, 该节点就越 重要. 介数中心性的计算时间复杂度较高, 使其在实 际应用中受到限制, 相关讨论可参见文献[].

\subsection{流介数中心性}
介数中心性仅考虑网络流通过最短路径传输. Yan 等人[]的研究指出, 如果选择最短路径来运输 网络流, 很多情况下反而会延长出行时间、降低出行 效率. 把一对节点之间的每条路径看作一条单独的 管道, 一条管道能够传输一个单位的网络流, 从源节 点 vs 到目标节点 vt 的最大流量是指 vs 与 vt 之间所有 管道可同时运输的网络流的总和(实际上, 这种假设 没有实际意义, 多条路径往往有重合的部分, 重合部 分的流量就会超过假设的情况). 基于这样的假设, 流介数中心性(flow betweenness centrality)[]认为网 络中所有不重复的路径中, 经过一个节点的路径的 比例越大, 这个节点就越重要. 由此得到节点 vi 的流 介数中心性为

			插入公式

			介数中心性和流介数中心性考虑的是两个极端, 前 者只考虑最短路径, 后者考虑所有路径并认为每条 路径作用相同, 接下来介绍两种介于两者之间的介 数中心性算法.

\subsection{随机游走介数中心性}
	从源节点 v 到目标节点 v 的随机游走的过程中 当i=s或者t的时候,IsIt1.该方法计算复杂度st stst 经过 vi 的次数可表征 vi 的重要性. 基于此, Newman[]提出了基于随机游走的介数中心性算法(random walk betweenness centrality). 在随机游走过程中短的路径 计数次数较多, 相当于赋予其更高的权重. 在随机游 走过程中, 如果网络流不断地从一个节点来回经过 无疑会提高这个节点的介数中心性, 但是这样的刻 画实际上是毫无意义的. 为了避免这种偏差, 约定在 一次随机游走中如果网络流两次分别从相反方向经 过某一节点, 则它们对这个节点的介数中心性的贡 献相互抵消. 于是, 节点 vi 的随机游走介数中心性可 表示为

			公式
\subsection{路由介数中心性}
计算机网络中, 每个路由器都有一个包含很多行记录的路由表, 每行记录存储着要到达的目标地址及下一跳地址. 显然, 每个路由器只记录了局部的网络结构信息. 对网络中的每一对节点(vs, vt), 将分布在各个路由器中的信息聚合, 可形成一个关于这一对节点的有向无环图 R(s, t). 定义 p(s, u, v, t)为有向无环图 R(s, t)中节点 vu 转发给节点 vv 一个从源节点vs 到目标节点 vt 的信息包的概率. 如果 p(s, u, v, t)>0,则在 R(s, t)中存在一条从 vu 指向 vv 的有向边. 用s,t(u)其中 I i程中, 经过节点 vi 的次数. 事实上, 如果我们将网络 看成一个电阻网络, 每条边的电阻值为 1, 从节点 vs表示信息包从 v 到 v 的传递过程中, 经过节点的 v stst st u概率, 显然s,t(s)=s,t(t)=1, 用 Preds,t(v)表示 R(s, t)中 节点 vv 的直接前驱的集合, 那么有向无环图 R(s, t)中 经过任意一个节点 vv 的概率可由下式得出:

gon shi
\subsection{子图中心性}
我们考虑经过节点的路径为一个封闭环的时候, 就可以定义子图中心性(subgraph centrality)[].该方法从全局的视野考察了网络中所有可达的邻居对节点中心性的增强作用, 并且认为增强作用会随距离的增加而衰减. 与图论中的概念有所不同, 这里一个子图特指从一个节点开始到这个节点结束的一条闭环回路. 一个节点 vi 的子图数目就是以该节点为首尾的闭环回路的个数. 子图中心性认为闭环回路的路径长度越小, 回路信息交流越便利, 节点之间的联系越紧密, 对节点的中心性贡献越大, 其定义为

			公式

			其中at为网络的邻接矩阵A的t次幂的第i个对角线元素.t=1时, a1 0;t=2时, a2为节点v 的度值, 即 ii ii ia2  k , 此时, 子图中心性就等价于度中心性; t3 ii i时, at表示从点v开始,经过t条边又回到v的路径 iii i的数目. 子图中心性赋予较短的回路较高的权重, 使 得节点的度在其中发挥较大作用的同时, 还考虑了 高阶回路[]. 在实际应用时, 根据具体计算需求, t 可 以取到任意值截断. 子图中心性用邻接矩阵特征值 和特征向量可表示为

			公式

			其中,jj1,2,,n为邻接矩阵A的特征值,j是  所对应的特征向量,  i 表示特征向量的第 i 个元素. 有些情况下, 度中心性, 接近中心性以及介数 中心性都不能区分网络中某些节点谁更重要时, 可 用子图中心性来对这些节点进行更加细致地区分[]. 另外, 子图中心性的方法还能够应用于网络中模体 的检测[].

\section{基于特征向量的排序方法}
前面介绍的方法都是从邻居的数量上考虑对节 点重要性的影响, 基于特征向量的方法不仅考虑节 点邻居数量还考虑了其质量对节点重要性的影响. 下面将详细介绍 7 种方法. 其中前两种方法, 即特征 向量中心性和累计提名方法一般用在无向网络中, 后者收敛更快. 后面五种方法可看成特征向量中心性 在有向网络中的应用. PageRank 算法和 LeaderRank 算 法通过模拟用户上网浏览网页的过程, 使节点的分 值沿着访问路径增加, 用于识别网页重要性. 实验结 果显示, LeaderRank 表现好于 PageRank 算法. HITs 算法、自动信息汇集算法, SALSA 算法中考虑节点的 双重角色: 权威性和枢纽性, 并认为两者相互影响. 本类方法在理论和商业上都受到了极大的关注, 很 有借鉴意义.
\subsection{特征向量中心性}
特征向量中心性(eigenvector centrality)[]认为一 个节点的重要性既取决于其邻居节点的数量(即该节 点的度), 也取决于每个邻居节点的重要性. 记 xi 为 节点 v 的重要性度量值, 则:

			公式

			特征向量中心性更 加强调节点所处的周围环境(节点的邻居数量和质 量), 它的本质是一个节点的分值是它的邻居的分值 之和, 节点可以通过连接很多其他重要的节点来提 升自身的重要性, 分值比较高的节点要么和大量一 般节点相连, 要么和少量其他高分值的节点相连. 从 传播的角度看, 特征向量中心性适合于描述节点的 长期影响力, 如在疾病传播、谣言扩散中, 一个节点 的 EC 分值较大说明该节点距离传染源更近的可能性 越大, 是需要防范的关键节点[].
			特征向量法完全用与某节点相连接的其他节点 的信息来评价该节点的重要性. Bonacich 等人[]认为 节点的重要性还可能受到不依赖于节点连接信息的 一些来自外部的信息的影响. 例如在微博上有人喜 爱转发其他人发布的信息(依赖于网络连接的内部信 息), 有的人却比较热衷于发布原创信息或从其他网 站转发一些信息(不依赖于网络连接的外部信息). 由 此 Bonacich 等人提出阿尔法中心性(Alpha-centrality), 即 x=Ax+e, 其中 为刻画来自网络内部连接影响的 内因参数, e 为刻画那些不受网络连接影响的外因参 数. 不失一般性,e可以设置为一个所有元素都等于1 的向量, 此时阿尔法中心性与 Katz 中心性一致.
			当网络中有一些度特别大的节点的时候, 特征 向量中心性会出现分数局于化现象(Localiztion), 即 大多数分值都集中在大度节点上, 使得其他节点的 分值区分度很低. 为了避免这一现象, Martin 等人[] 对特征向量中心性进行改进, 提出在计算节点vi的分 值时, 求和中其邻居的分值不再考虑节点vi的影响.
\subsection{累计提名}
特征向量中心性中, 一个节点的打分值完全由邻居决定, 收敛过程缓慢. 此外, 当不存在一个正的自然数 t, 使得转移矩阵的 t 次幂所有元素都是正的时, 节点打分值会出现周期性循环, 不能收敛. 为了使打分值能够收敛并且快速收敛, 累计提名(cumulative nomination) []方法在每次迭代过程中,同时考虑邻居节点和自身的打分值. 设 pit 为节点 vi在时刻 t 时得到的提名次数, 假设 t=0 时每个节点都获得 1 次提名(即 p 0  1 ), 每个时间步每个节点从所 i有相邻的节点处获得新增的提名, 新增的提名数为 邻居节点已有的提名数的总和. 于是定义节点在 t+1 时刻的累积提名为

			公式

			如果所有节点归一化后的提名次数不再变化, 则停 止迭代. 稳态时每个节点的提名次数占所有节点的 提名次数的比例就是其重要性权值. 特征向量中心 性算法在每次迭代的时候, 一个节点 vi 的中心性值 完全等于邻居的中心性值之和, 而累计提名算法则 保留了节点 vi 上一步的中心性值, 实验结果显示累 积提名相比原始的特征向量中心性收敛速度更快. 累积提名和 Alpha 中心性在数学形式上非常相似, 但 Alpha 中心性中的 e 是固定值, 即每次迭代的时候不 变, 而累积提名中添加的是上一时间步的打分值, 这 个打分值会随着每步更新变化.
\subsection{PageRank 算法}
特征向量中心性及其变体应用广泛, 例如网页 排序领域中最著名的 PageRank 算法[], 是谷歌搜索 引擎的核心算法. 传统的根据关键字密度判定网页 重要程度的方法容易受到“恶意关键字”行为的诱导, 使搜索结果可信度低. PageRank 算法基于网页的链 接结构给网页排序, 它认为万维网中一个页面的重 要性取决于指向它的其他页面的数量和质量, 如果 一个页面被很多高质量页面指向, 则这个页面的质 量也高. 初始时刻, 赋予每个节点(网页)相同的 PR 值, 然后进行迭代, 每一步把每个节点当前的 PR 值 平分给它所指向的所有节点. 每个节点的新 PR 值为 它所获得的 PR 值之和, 于是得到节点 vi 在 t 时刻的 PR 值为

			公式

			jj以 c 的概率均分给网络中所有节点, 以 1c 的概率均 分给它指向的节点. 该过程实际上是考虑到了现实 中网络用户除了通过超链接访问页面之外, 还可以 通过直接输入网址的形式对网页进行访问的行为, 从而保证了即使是没有任何入度的网页也有机会被 访问到. 其实质是将有向网络变成强连通的, 使邻接 矩阵成为不可约矩阵, 保证了特征值 1 的存在. 由此 可得含参数 c 的 PageRank 算法:值都达到稳定时为止. 公式(27)的缺陷在于 PR 值一 旦到达某个出度为零的节点(称为悬挂节点 Dangling node), 就会永远停留在该节点处而无法传递出来, 从而不断吸收 PR 值[]. 为解决这一问题, PageRank 算法在上述过程基础上引入一个随机跳转概率 c. 每 一步, 不管一个节点是否为悬挂节点, 其 PR 值都将

			公式 

			参数 c 的取值要视具体的情况而定. c 取值越大收敛越快, c=0 时回到公式(27). c 取值越大算法的有 效性越低, c=1 时所有节点都有相同的 PR 值. 针对万 维网的网页排序, 以前的研究显示, c=0.15 是一个比 较好的参数. PageRank 算法作为谷歌搜索引擎的核 心算法, 它在商业应用上的极大成功激发了人们深 入研究 PageRank 的热忱, 研究者们提出了一系列基 于 PageRank 的改进算法. 例如 Kim 和 Lee[]为了避 免悬挂节点囤积 PR 值的问题, 将每一步到达悬挂节 点的 PR 值平均分给网络中的 n 个节点, 即将概率转 移矩阵中悬挂节点所在的列的 n 个元素修改为 1/n; PageRank 中从一个网页上的链接中挑选下一个访问 目标时是等概率的, Zhang 等人[]认为这 n 个目标网 页出度越大的越有可能被点击, 并提出 N-step PageRank 算法用以描述这一思想. 2012 年 Brin 和 Page[]以相同的题目重新出版了当年提出 PageRank 算法的博士学位论文, 在文中他们对这十几年的网 页排序算法进行了回顾, 并就如何用 PageRank 实现 大规模搜索进行了深入讨论. 另外, 作为有向网络节 点排序最经典的算法, PageRank 及其改进算法广泛 应用于其他领域, 如对期刊的排序[]、对社交网络上 用户的排序[]、对风投公司(VC)的排序[]、对科学论 文的排序[]~73]以及科学家影响力的排序[]~77]等.
\subsection{LeaderRank 算法}
PageRank 算法中, 每一个节点的随机跳转概率 都是相同的, 即从任意网页出发, 采用输入网址来访 问其他网页的概率相等. 然而在现实中人们在内容 丰富的热门网页(出度大的节点)上浏览的时候选择 使用地址栏跳转页面的概率要远小于浏览信息量少 的枯燥网页(出度小的节点). 另一方面, PageRank 算 法中的参数 c 的选取往往需要实验获得, 并且在不同的应用背景下最优参数不具有普适性[]. LeaderRank 算法的出现很好地解决了以上两个问题. 在有向网 络的随机游走过程中, 通过添加一个背景节点以及 该节点与网络中所有节点的双向边来代替 PageRank 算法中的跳转概率 c, 从而得到一个无参数且形式上 更加简单优美的算法. LeaderRank 算法在某一页面输 入网址访问下一个页面的概率就相当于从这个页面 访问背景节点的概率, 这个概率和一个网页上的链 接数负相关, 链接数越多, 网页的内容越丰富, 越倾 向于从本地的链接访问, 访问背景节点的概率就越 低. 注意, 背景节点的存在同样保证了网络的强连通 性. 初始时刻给定网络中除背景节点 vg 以外的其他 节点单位资源, 即LRi01,ig; LRg00. 经过以下的迭代过程直到稳态:

			公式

			LeaderRank 算法在衡量社会网络中节点的影响力等方面有非常优异的表现[], 因此得名. 实验发现 LeaderRank 比 PageRank 在很多方面表现得更好:(1) 与 PageRank 相比收敛更快[]; (2) 能够更好地识别网络中有影响力的节点, 挖掘出的重要节点能够将网络流传播的更快更广; (3) 它在抵抗垃圾用户攻击和随机干扰方面相比 PageRank 有更强的鲁棒性.这些优点使得 LeaderRank 算法广受关注. 标准LeaderRank 算法中背景节点和所有节点的连接都一样, Li 等人[]对此提出改进, 认为从背景节点出发访问其他节点时, 入度大的节点应该有更高的概率被访问到. 如果一个节点 v 的入度为 kin , 则背景节点 ii指向v的边权w (kin), 网络其他节点之间的连接 i gi i的权重都等于1, 由此得到改进后的LeaderRank的迭 代公式为

			公式

			这种改进更加重视网络中的大度节点, 在多个数据集上的实验发现新方法比标准的 LeaderRank 的性能 在多个方面均有提升. 虽然这一方法的提出最初是为了提升 LeaderRank 算法在无权网络中的排序效果, 但是这种思路也可以应用到含权网络中, 关于 LeaderRank 算法在含权网络中的扩展参见 5.5 节
\subsection{HITs 算法}
一个网络中不同类型的节点功能不同, 每个节 点的重要性往往不能由单独的一个指标给出, HITs 算 法[]赋予每个节点两个度量值: 权威值(Authorities) 和枢纽值(Hubs). 权威值衡量节点对信息的原创性, 枢纽值反映了节点在信息传播中的作用. 枢纽页面 是那些指向权威页面的、链接数较多的页面, 反映网 页上链接的价值. 节点的权威值等于所有指向该节 点的网页的枢纽值之和, 节点的枢纽值等于该节点 指向的所有节点的权威值之和. 因而, 节点若有高权 威值则应被很多枢纽节点关注, 节点若有高枢纽值 则应指向很多权威节点. 简单地说, 权威值受到枢纽 值的影响, 枢纽值又受到权威值的影响, 最终通过迭 代达到收敛.
			在一个包含 n 个节点的网络中, 定义 ait 和 hit 分别 为节点 vi 在时刻 t 的权威值和枢纽值, 于是在每一时 间步的迭代中:

			公式

			HITs 首次用不同指标同时对网络中的节点进行 排序, 具有开创意义. HITs 除了可以用于确定一个节 点上多个相互关联的属性, 还可以处理更复杂的排 序问题[], 譬如在信誉评价系统中如何评价用户 的信誉度以及产品的质量[]. 这类评价系统通常包 含两类节点(用户和产品), 信誉排序问题解决的是包 含两类节点的各自的排序问题. 与 HITs 类似的是两 类节点的分数值也是相互影响的, 最终通过迭代寻 优获得两类节点的排序值. 例如文献[]利用这种思 路提出一种可以有效抵抗恶意评分的排序方法, 该 方法认为一个商品得到的打分反映了这个商品的质 量, 自然地, 应该给可信度高的用户更大的权重; 反 过来, 一个用户打分的可信度, 可以用他的打分和商 品质量的接近程度来衡量.

			需要指出的是, 特殊的网络结构会影响 HITs 算 法、PageRank 算法这类应用邻居之间相互传递打分值进行排序的方法的表现. 例如万维网中广泛存在 紧密连接社团(tightly-knit community), 社团内节点 间非常紧密的链接关系会使这些节点的权威值和枢 纽值相互增强(mutual reinforcement), 从而使网页的 排序结果更倾向于将社团内部的页面排在前面而偏 离搜索的主题, 出现主题漂移(topic draft)现象[].

\subsection{自动信息汇集算法}
Kleinberg 与其合作者对 HITs 算法进行了改进,提出了自动信息汇集(automatic resource compilation,ARC)算法[]. HITs 算法仅考虑网页之间的链接关系(即仅考虑网络结构), ARC 算法在此基础上, 还考虑了页面内容与搜索主题的相关性, 给每个链接赋予不同的权值, 提高页面排序的真实可靠性. 算法的具体过程如下: 取一个含有搜索主题 T 的网页的增广集, 这个集合中的网页抽象为节点, 它们之间的链接抽象为节点之间的连边. 每个节点 vi 都有权威值 ai和枢纽值 hi, 所有节点的初始权威值设为 1. 假设某一个页面上有一个指向另一个页面的链接, 如果链接周围有较多关于搜索主题 T 的内容, 则认为链接的权值较大. 记 t 为链接前后 B 字节范围内关于主题T 的内容出现的次数, 定义链接的权值 w =t+1, 在每 ij步迭代之后进行归一化. 作者提出 ARC 算法时建议 B=50. 接下来, 通过下面的迭代过程使权威值和枢 纽值达到稳定:

			公式

			与此类似, 文献[]也提出一种同时考虑页面之 间的链接和页面内容的排序算法, 与 ARC 不同的是它对页面内容采用的是语义分析技术.
\subsection{SALSA 算法}
SALSA 算法[], 即链接结构的随机分析法(stoch- astic approach for link structure analysis), 是 HITs 算 法的另一种改进. SALSA 算法不仅考虑了用户在浏 览网页时顺着网页之间的链接方向访问网页, 还考 虑了逆着链接方向访问原来的网页的情况. SALSA 算法用随机游走的方法, 通过访问网页的马尔科夫 过程来确定网页的权威值和枢纽值的大小. 万维网 用有向网络 G 表示, 所有入度不为零的节点构成权 威集合 SA, 所有出度不为零的节点构成枢纽集合 SH, 两类节点之间的关系用无向边来表示: 图 G 中从节 点 vi 指向 vj 的边表示为边(iH, jA), 由此将原始网络 G转换为无向二分网络 G , 图 2 给出一个示例.
			用 G 中长度为 2 的路径模拟用户上网的随机游 走过程, 则每一个随机游走的路径都是从集合 SA 到 集合 SH 再到集合 SA 或从集合 SH 到集合 SA 再到集合 SH, 其中每一个从集合 SH 到集合 SA 的路径都是沿着 链接方向访问, 每一个从集合 SA 到集合 SH 的路径都 表示逆着链接方向访问. 每一随机游走过后节点上 的权值都会进行重新分配. 于是可以根据枢纽值和 权威值定义两个随机游走过程. 对于计算枢纽值而 言, 初始时刻赋予枢纽集合中的每个节点一单位初 始权值, 用向量 h0 表示, 权值转换的过程可表示为, 其中权值转换矩阵H的元素(H)ij h
			其中 ai 为二分网络 G 的邻接矩阵元素, 如果节点 vi (SA )与节点 v (SH )相连接则ai 1, 否则 ai0.kc表示二分图中节点vc的度,当vcSA时kc 相当于节点vc在图G中的入度,当vcSH时kc相当 于节点 vc 在图 G 中的出度. 类似地, 权威值的转换过 程为atAat1,其中转移矩阵A的元素(A) a  表示节点 v 将其权威值传给节点 v 的概率, 即:
			多次迭代后每个节点上的值都达到稳定时停止 迭代, 于是得到节点最终的权威值和枢纽值. 由于计 算枢纽值和权威值的随机过程是相互独立的, 因此 不会出现两者相互增强的情况, 相比 HITs 算法而 言, SALSA 算法能够更好地避免主题漂移的问题. SALSA 算法实际上考虑的是一个基于二部分图的随机游走过程, 这一思路也被成功地应用在信息挖掘 的另外两个领域中, 即基于网络结构的链路预测问 题[]和个性化推荐算法[]. 实际上这里介绍的 SALSA 算法和推荐算法中的物质扩散算法如出一 辙[], 其区别在于以下几点: (1) 推荐系统中的物 质扩散算法通常只考虑扩散两步的结果, 并不考虑 稳态的结果; (2) 在个性化推荐中初始向量的设定根 据目标用户不同而异, 而 SALSA 算法不会针对某一 个节点设置不同的初始向量值; (3) 在推荐算法中通 常只考虑用户没有选择过的产品的排序结果, 而 SALSA 考虑的是对所有节点的排序结果.

\section{基于节点移除和收缩的排序方法}
节点(集)的移除和收缩方法与系统科学中确定[] 一个系统的核心的思路暗合 , 其最显著的特点是在重要节点排序的过程中, 网络的结构会处于动态 变化之中, 节点的重要性往往体现在该节点被移除 之后对网络的破坏性. 从衡量网络的健壮性角度看, 一些节点一旦失效或移除, 网络就有可能陷入瘫痪 或者分化为若干个不连通的子网. 实际生活中的很 多基础设施网络, 如输电网、交通运输网、自来水- 天然气供应网络等, 都存在“一点故障, 全网瘫痪”的 风险. 为了预防风险, 研究人员提出了很多方法来研 究节点收缩或者移除之后网络的结构与功能的变化, 从而为新系统的设计与建造提供依据. 比较典型的 是系统的“核与核度”理论. 许进等人[]在定义规则 网络图的核概念基础上, 提出了核度的测量方法, 研 究了网络核度与节点数、边数的关系, 并根据它们之 间的关系设计了规则网络构造定理; 李鹏翔等人[] 认为直接的联系往往是间接联系的必经之路, 在评 估节点重要性的过程中更加重要, 用节点集被删除 后形成的所有不直接相连的节点对之间的最短距离 的倒数之和来反映节点删除对网络连通的破坏程度; 陈勇等人[]分析了通信网络, 考察去掉节点(集)及其 相关边后所得到的图的生成树的数目, 数目越小, 表 明该节点(集)越重要; 谭跃进等人[]用收缩节点方法 替代删除节点法, 综合考虑了节点的度以及经过该 节点的最短路径的数目, 将节点收缩后网络的聚集 度作为节点重要性评估的标准. 系统科学的方法给 我们提供了新的视角, 但由于计算复杂度较高, 目前 这类方法还仅限于小规模的网络实验. 此外, Restrepo 等人[]提出通过考察网络最大特征值在移除节点后的变化来衡量节点重要性的方法, 该方法还可以应 用于刻画网络连边的重要性.
\subsection{节点删除的最短距离法}
破坏性反映重要性. 节点删除的最短距离法[] 认为一个节点移除后的破坏性与所引起的距离变化 有关: 移除一个节点(集)会引起网络分化, 并形成若 干个连通分支, 网络中节点对之间较短距离的变化 越大, 被移除的节点就越重要. 该算法区别对待不同 长度的路径, 认为“相对直接的、近距离的联系所造 成的破坏性大于相对间接的、远距离的联系所造成的 破坏性”[]. 具体地, 在连通图中一个节点被删除之 后, 对网络的整体状况的影响体现在两个方面: 直接 损失和间接损失.
			直接损失是指被删除的节点与其他剩余的节点之间不再存在通路, 如果连通网络中共有 n 个节点,删除一个节点后产生的不连通节点对的数目为 n1.如果删除的是节点集, 直接损失还应该包括删除的节点集内节点之间的不再连接的损失. 间接损失是指删除一个节点造成剩余节点之间不连通而引发的损失: 用 Nk (k=1, 2, ···, s)表示一个节点 vi 被删除后,网络分化成的 s 个连通子图中第 k 个连通子图的节点数, 则该节点被删除后所形成的不再连通的节点对的数目为s s N N , 记由于删除节点 v 造成 t1 rt1 t r i的不再相连的节点对表示为集合 E (包括直接损失和 间接损失两部分), 那么节点 vi 的重要性等于集合 E 中节点对之间的最短距离的倒数之和, 即:

			公式

			djk为删除节点vi之前vj与vk间的最短距离. 注意, 当 j或k=i的时候, 相当于直接损失; 当 jki的时候, 相当于间接损失. 节点删除的最短距离法在衡量一 些节点集的重要性方面优势比较突出. 在实际的大 规模网络中, 仅删除一个节点时网络的拓扑图一般 不会分化为几个连通子图, 网络的间接损失为 0, 节 点删除的最短距离法效果并不明显. 而如果同时删 除多个节点, 则很容易使网络不再连通, 这时该方法 的优越性就显现出来了.

\subsection{节点删除的生成树法}
在通信网络中, 节点删除后网络中节点对之间最短距离会发生变化, 但一般对网络时延影响不大,用最短距离法不一定准确. 这时可通过考察节点删 除后网络拓扑图的生成树个数来衡量节点的重要性. 在图论中, 一个图的树是该图的一个连通的无环子 图, 一个图的生成树定义为拥有该图的所有顶点的 树. 节点删除的生成树法[]认为一个节点删除后对 应的网络的生成树的数目越少, 该节点越重要. 给定 一个无向连通图, 其邻接矩阵为 A, 网络拉普拉斯矩 阵 L=D–A(将矩阵 A 主对角线上的元素 aii 替换为节点 vi 的度值, 非对角线上的元素值全部乘以1). 那么, 这个连通无向图的生成树个数 t0 为矩阵 L 的任意一 个元素lpq的余子式Mpq的行列式, 即: t0  Mpq . 删除任意一个节点vi, 网络的邻接矩阵变为Ai, 然后用 上面的方法计算网络的生成树个数为 ti. 由此可定 义节点vi 的中心性指标为

			公式

			在节点的移除对网络的连通性影响不大的网络 中, 节点删除的生成树法优于最短距离法. 但节点删 除的生成树法有一些缺点, 例如, 只能用在连通网络 中. 若一个节点删除后网络变得不再连通, 这些节点 的重要性就难以判断了, 这时可采用节点收缩法评 估节点的重要性.
\subsection{节点收缩法}
节点收缩就是将一个节点和它的邻节点收缩成 一个新节点[]. 如果 vi 是一个很重要的核心节点, 将 它收缩后整个网络将能更好地凝聚在一起. 最典型 的就是星形网络的核心节点收缩后, 整个网络就会 凝聚为一个大节点. 从社会学的角度讲, 社交网络中 人员之间联系越方便(平均最短路径长度 d 越小), 人 数越少(节点数 n 越小), 网络的凝聚程度就越高. 因 此定义网络的凝聚度为

			公式

			可见, 节点收缩法中节点的重要程度由节点的邻居数量和节点在网络路径中的位置共同决定. 由于每 次收缩一个节点, 都要计算一次网络的平均路径长度, 时间复杂度比较高, 不适于计算大规模网络.

\subsection{残余度的中心性}
为了研究网络的抗毁性, Dangalchev[]提出了残 余接近中心性(residual closeness centrality), 用来衡 量节点的移除对网络带来的影响. 残余接近中心性 认为若一个节点的删除使得网络变得更加脆弱, 该 节点就越重要. 文献[]对接近中心性的改进使得接 近中心性应用的范围从连通图扩展到了非连通图. 该方法对接近中心性进行了改进, 分母取以2为底的 指数, 相当于提升了短路径的影响力, 同时会使本算 法更易计算和扩展(文献[]给出了将几个图合并为 一个图计算接近中心性的详细算法). 在移除一个节 点 vi 之后, 定义其残余接近中心性为

			公式

			其中 djk(i)为删除节点 vi 之后, 节点 vj 与 vk 的最短距离. 残余接近中心性在测度网络的脆弱性方面比图 坚韧度(graph toughness)、离散数(scattering number)、 节点完整度(vertex intergrity)1)等方法表现要好. 基于 该方法可以定义出边的残余接近中心性和节点集、边 集的残余接近中心性.

\section{权网络中的节点中心性}
无权网络采用粗粒化的二分法来表示网络中节 点间的联系(有边为 1, 无边为 0), 不考虑联系的强弱 信息. 然而边的权重信息能帮助我们更加细致地理 解网络的结构与功能. 如在社交网络中, 边的权值可 代表情感关系的强弱、交流与服务的频次、任务执行 时间的长短等. 科学家合作网络中可用两个科学家合作论文的数量刻画两个科学家的联系紧密性. 航 空运输网络中可以用两个机场之间所有班次上的座 位数表示这两个机场的通勤情况. 那么, 如何能够有 效地利用网络的边权重信息进行重要节点的挖掘呢? 到目前为止, 大多数的研究思路都是将基于无权网 络中心性指标在含权网络上进行扩展应用, 专门针 对含权网络进行设计的方法鲜见.



\section{节点重要性排序方法的评价标准}
根据评价标准的不同又分为用网络的鲁棒性和脆弱性评价排序算法、用传播动力学模型评价排序算法。
		网络科学研究的早期, 所关注的网络中节点数 目较少, 典型的有同性恋接触网络[]、女生用餐伙 伴选择网络[]、空手道俱乐部网络[]等, 对于这些 小规模网络, 可以通过调查问卷等方式对每个节点 的重要性进行打分, 然后将实际的调查结果作为标 准与其他算法结果进行比较, 分析各种方法的表现 和优劣. 随着科技的发展和进步, 大数据时代已经来 临, 现在我们所面对的网络规模迅速增长, 想要得到 一个对所有节点的重要性的较为客观的评价标准极 为困难. 目前评价各种排序算法优劣的主要思路是: 将排序算法得出的重要节点作为研究对象, 通过考 察这些节点对网络某种结构和功能的影响程度、对其他节点状态的影响程度来判断排序是否恰当. 例如, 如果一个排序算法得出节点 vi 比 vj 更重要, 单独考察 vi 比 vj 发现前者对网络的结构功能或对其他节点的影 响程度更大, 就说明这种排序算法比较符合实际. 常 用来评价各排序算法的方法有基于网络的鲁棒性和 脆弱性方法以及基于网络的传播动力学模型的方法. 下面分别对这两类方法进行简单的介绍.
\subsection{用网络的鲁棒性和脆弱性评价排序算法}
本类方法着重考察网络中一部分节点移除后网络结构和功能的变化, 变化越大移除的节点越重要.用某一种重要节点挖掘方法将网络中所有节点按重要性进行排序, 然后按重要性从大到小的顺序, 将一部分节点从网络中移除, 用(i/n)表示移除 i/n 比例的节点后, 网络中属于巨片(giant component)[]的节点数目的比例, 网络的鲁棒性(robustness)可用 R-指 标刻画[]:

			公式

			显然, 不论对何种算法, 星形图中, R 取最小值 (1/n1/n2), 完全图中 R 取最大值(11/n)/2, 当 n 比较 大时R0,1/2. 可定义V12R来表示网络对于 所实施的移除方法的脆弱性(vulnerability), 可见, V- 指标越大表示采用该方法进行攻击的效果越好. V-指 标和 R-指标可从整体上反应各种重要节点挖掘方法 的有效性. 另外也可画出i/n与(i/n)在二维坐标上的 曲线, 对节点移除的影响进行详细分析. 例如文 献[]中考察了在无标度网络中使用 4 种排序方法 移除节点后对网络最大连通集的影响, 这4种方法包 括度中心性、介数中心性、接近中心性和特征向量中 心性, 并和随机移除节点的方法进行比较. 用于实验 的无标度网络节点数为 n=10000, 平均度为 4 (图 4(a)) 和 6(图 4(b)), 移除节点时采用同时移除的方法.
\subsection{用传播动力学模型评价排序算法}
复杂网络上传播研究的对象极广[], 比如通 信网络中的病毒传播[]、社会网络中的信息传 播[]、电力网络中的相继故障[]、经济网络中的危 机扩散等[]. 在评价各种节点重要性挖掘方法时广 泛采用的是传染病模型, 主要包括 SIS 模型[]和 SIR 模型[]. 在 SIS 模型中一个节点的传播能力被定义 为稳态下该节点被感染的概率; 在 SIR 模型中, 一个节点的传播能力被定义为该节点的平均传播范围.
			下面简要介绍 SIR 模型及一个应用的例子. SIR 假设网络中的节点有三个状态: 易染态 S (susceptible, 可被处于感染态的邻节点感染), 感染态 I (infected, 处于 I 态的节点一定时间后会变为免疫态), 免疫态 R (recovered, 免疫态的节点不会被感染, 也不会传播 病毒). SIR 模型有单点接触和全接触两种[], 前者 指在每一时间步内, 处于 I 态的节点感染其邻居的时 候将随机选择一个 S 态的邻居, 然后以概率 p 使其由 S 态变为 I 态; 后者指处于 I 状态的节点感染邻居的 时候选择的是所有 S 态的邻居, 每个 S 状态的邻居都 有机会以概率 p 转变为 I 态. 设置一个(组)节点为初 始感染节点(即处于 I 态), 观察每一时间步网络中感 染过的节点数目和最终稳定态时(没有 I 态的节点时) 感染过的节点数目, 可通过病毒的传播速度和范围 两个方面来考察节点的真实影响力. 要对比两种重 要节点挖掘方法的优劣, 可分别用这两种方法对网 络中的节点按重要性进行排序, 取相同数目的最重 要的节点设为初始感染态, 用SIR模型在网络上进行 实验, 如果一个排序方法的结果使得网络流传播地又快又广, 则说明该重要节点排序方法优于其他方 法. 例如文献[]中应用 SIR 模型比较了 LeaderRank 算法和 PageRank 算法的排序结果. 图 5 显示了使用 两种方法获得的前 20 个(图 5(a))最重要的节点中, 以 不同的节点为初始感染源进行 SIR 传播的过程. 可见, 以 LeaderRank 获得的节点为初始感染源的传播又快 又广, 说明 LeaderRank 算法比 PageRank 算法更能够 识别网络中传播影响力高的节点. 图 5(b)为考虑前 50 个节点的情况.需要注意的是, 网络中信息传播和病毒传播有 很大的不同. 文献[]深入比较了信息传播与病毒 传播的不同, 提出了网络中的信息传播模型. 文中还 全面总结了影响网络流在网络中传播速度和快慢的 7种因素,比如边的强度、信息内容、传播者的角色、 记忆效应、时间延迟效应等. 因此, 在评价节点信息 传播影响力的时候, 例如社交网站上意见领袖挖掘, 应该考虑更加符合实际传播方式的模型.
		